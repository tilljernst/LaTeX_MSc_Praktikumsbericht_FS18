\documentclass[jou,a4paper,apacite]{apa6}
\usepackage[utf8]{inputenc} % für die Umlaute

% Für Verweise innerhalb Doc (siehe https://strobelstefan.org/?p=145)
\usepackage{csquotes}
\usepackage{prettyref}
\usepackage{titleref}
%%% Für Abschnitte %%%
\newrefformat{sec}{siehe Abschnitt~\ref{#1} \enquote{\titleref{#1}} \ auf Seite \pageref{#1}}
%%% Für Abbildungen %%%
\newrefformat{fig}{siehe Abb.~\ref{#1} \enquote{\titleref{#1}} \ auf Seite \pageref{#1}}
%%% Für Tabellen %%%
\newrefformat{tab}{siehe Tab.~\ref{#1} \enquote{\titleref{#1}} \ auf Seite \pageref{#1}}


\title{Praktikumsbericht ER4 \& Intervisionsreflexion - Psychiatriezentrum Breitenau}
\shorttitle{Praktikumsbericht}

\author{Stud. MSc Till J. Ernst}
\affiliation{Applied Psychology ZHAW}

\abstract{Im folgenden Dokument soll aus persönlicher ...}

%\rightheader{APA style}
%\leftheader{Author One}

\begin{document}
\maketitle    

%#######################   
\section{Praktikumsbericht}\label{sec:Praktikumsbericht}
\subsection{Einleitung} \label{sec:Einleitung}
Im Rahmen der Ausbildung an der ZHAW zum Psychologen, wird im Masterstudiengang im Modul ER4  ein Praktikum von 300 Stunden vorgeschrieben. Am Ende der praktischen Erfahrung soll in Form einer schriftlichen Reflexion auf den Rahmen und den persönlichen Prozess während dieses Praktikums eingegangen werden. Dieser Praktikumsbericht ist Bestandteil des Moduls und trägt zur Bewertung bei. 

In einem ersten Teil werden auf die Rahmenbedingungen und Fakten des Praktikums eingegangen. Da soll der der Ort, die Praktikumsleitung, die Betreuung, die Supervision und die ausgeführten Tätigkeiten vorgestelt werden.

Ein einem zweiten Teil wird die persönliche Auseinandersetzung mit dem Praktikum erläutert. Darin sollen ausgeführte Funktionen wie die Gruppentherapie und die Psychoedukation im Detail angeschaut, persönliche Höhepunkte wie mein veränderter Umgang mit Menschen oder die Therapiegstaltung an der Klinik erläutert und weitere Punkte wie die Stellung der Psychologen und die Akzeptanz von Praktikanten aus meiner Sicht beleuchtet werden.






\subsection{Rahmenbedingungen und Fakten (TBD: 1 bis 2 Seiten)} \label{sec:Rahmenbedingungen}

%----------------------------------------------
\subsubsection{Ort, Dauer und Entlöhnung} 
Das Praktikum im Rahmen der Ausbildung an der ZHAW zum Psychologen führte ich an der \textit{Klinik für Psychiatrie und Psychotherapie} aus, welche als psychiatrischen Dienst der Spitäler Schaffhausen angegliedert ist. Die Klinik hat vom Kanton Schaffhausen einen Versorgungsauftrag zur stationären, teilstationären und ambulanten Behandlung und Betreuung erwachsener psychisch Kranker. Die Klinik verfügt über 60 Betten in den Akutpsychiatrie und Rehabilitation. Zudem sind 12 Plätze in der psychiatrischen Tagesklinik vorhanden. Das Gebiet für Versorgung umfasst einerseits den gesamten Kanton Schaffhausen, sowie die angrenzenden Gebiete. Die Klinik arbeitet eng mit weiteren Kliniken des Kantons Zürich zusammen und nimmt bei Bedarf (wenn die angeschlossenen Kliniken keine Betten mehr haben) Patienten aus Zürich auf.

Das Praktikum dauerte vom 3.1.2018 bis und mit 30.3.2018 und umfasste insgesamt 433,5 Stunden zu 8,5 Stunden pro Tag bei einer Anstellung von 100\% (gemäss Vorgaben sind im Masterstudiengang 300h vorgesehen). 

Die Entlöhnung bei 100\% umfasste pro Monat CHF 1000.- (brutto) und wurde per Ende Monat auf das Lohnkonto überwiesen. Zusätzlich wurde ein Teil des 13ten Monatslohns Ende März überwiesen. Das Praktikum entsprach einer herkömmliche Anstellung am Spital. Aus diesem Grund wurden mit in dieser Zeit 5,5 Tage Ferien zur Verfügung gestellt. Zudem wurde ich an den in diesem Semester fälligen Unterrichtstagen an der ZHAW freigestellt. 

Für diesen drei Monaten wurde ich fest auf der Akutstation für affektive Störungen und Persönlichkeitsstörungen eingeplant. Diese Station hatte die Bezeichnung A2 und besitzt eine Kapazität von rund 20 Betten. Zudem verfügt diese Station über einen Isolationsraum, der bei Bedarf eingesetzt werden kann. Die restlichen Zimmer sind in Zwei- und Dreibettzimmer aufgeteilt, wobei die Kapazität im Notfall mit sogenannten Überbetten erhöht werden kann. 

In den drei Monaten hatte ich die Möglichkeit neben der Station A2 weitere Stationen in Form einer Tageshospitation zu besuchen. Ich nahm von diesem Angebot gebrauch und verbrachte je einen Tag auf der Station für psychotische Störungen und in der Tagesklinik. 

%----------------------------------------------
\subsubsection{Praktikumsleitung (Person / Funktion)}
Im Psychiatriezentrum gab es bis zu meinem Austritt kein Einzelperson, die für die Praktikumsleitung verantwortlich war. Die Leitung wurde anhand unterschiedlicher Teilbereichen unter verschiedenen Personen aufgeteilt: Für die zeitliche Planung und Auswahl der Praktikanten und Praktikantinnen zeichnete sich Frau Sybille Echte-Schnauber verantwortlich, Fachpsychologin Psychologischer Dienst Psychiatrie. Sie war Ansprechperson für die Bewerbung und war für die Auswahl der Psychologiepraktikantinnen und Praktikanten zuständig. Für den Inhalt des Praktikums zeichnete sich Herr Dr. biol. hum. Bernd Lehle verantwortlich. Aktuell leitender Psychologe der Tagesklinik. 

Diese beiden oben erwähnten Personen waren für die Auswahl und den groben Ablauf des Praktikums zuständig. Da alle Praktikantinnen und Praktikanten einer bestimmten Station angeschlossen sind, werden sie als angehende Psychologinnen und Psychologen automatisch dem zuständigen Oberarzt unterstellt. In meinem Fall war dies Herr Dr. med. Walter Brogiolo, Therapeutischer Bereichsleiter Akutpsychiatrie und Therapeutischer Leiter A2. Wie das in Kliniken Usus ist, möchte ich hier die oberste Instanz nicht unerwähnt lassen, welche durch den Chefarzt Herr PD Dr. med. Bernd Krämer abgedeckt wurde. Ich erwähne ihn aus dem einfachen Grund, da er bei Unstimmigkeiten im Verlauf eines Praktikums, welche nicht vom zuständigen Oberarzt, noch von der Praktikumsleitung gelöst werden kann, intervenierend einzugreifen vermag. Zudem konnte ich Herrn Krämer als an den Prkatikantinnen und Praktikanten interessierten Arzt kennenlernen, der auch gerne Anregungen und Tipps zum individuellen Praktikum abgegeben hat.

%----------------------------------------------
\subsubsection{Angaben über Betreuung, Super- und Intervision}
Die eigentliche Praktikumsanleitung wurde von Frau M.Sc. Martina Piraccini übernommen. Psychologin in Ausbildung auf der Akutstation A2 für affektive Störungen, auf welcher ich in den drei Monaten hauptsächlich angeschlossen war. Sie hatte die Funktion der ersten Ansprechperson und übernahm die Koordination der einzelnen Praktikumstätigkeiten. Diese Aufgabe hat sie mit der zuständigen Aktivierungstherapeutin, Frau Birgitta Busin, wahrgenommen, welche für meine Einbettung in die Pflege verantwortlich war. 

Einerseits führte das Pflegeteam monatlich eine Supervision durch. Zudem wurde monatlich eine Fallsupervision mit dem gesamte Team der Akutstation A2 (von der Pflege zum Oberarzt, über die Psychologinnen und Assistenzärztin der Station) durchgeführt. Diese Supervisionen wurden beide von einer externen Fachperson, Herr Thomas Disler, Eidgenössisch anerkannter Psychotherapeut und Diplomsupervisor, durchgeführt. An beiden Veranstaltungen durfte ich als Praktikant teilnehmen. 

Zudem wurde bei Bedarf ein spezifisch Coaching / Supervision für Praktikanten bei Frau Ronit Rüttimann, Fachpsychologin psychologischer Dienst, am Haus angeboten. Dieses Coaching nahm ich zu Beginn wöchentlich, danach etwas jede zweite Woche wahr. Dieses Coaching diente dazu, Fragen und Probleme im Zusammenhang mit dem Praktikum zu besprechen.

Durch die ZHAW wurde eine durch Studierenden geführte Intervision vorgeschrieben. Diese Intervision wurde von drei Mistudierenden und mir alle paar Wochen durchgeführt. Darin haben wir unsere individuellen Anliegen und Fälle gegenseitig vorgestellt und besprochen (\prettyref{sec:Intervision}). 

%----------------------------------------------
\subsubsection{Auflistung aller ausgeführter Funktionen}
Im Folgenden werde ich alle Funktionen auflisten, die ich in den drei Monaten durchführen konnte. In einem ersten Teil werde ich nur knapp auf diese eingehen und im zweiten Teil auf ausgewählte näher und ausführlicher eingehen.
\begin{itemize}
        \item Diverse Hospitationen unter anderem bei Einzelgesprächen mit Patienten, Oberarzt-Visiten, Chefarzt-Visiten, Morgen-Rapporte, Somatik-Rapporte, Interdisziplinären-Rapporten, Elternrunden, welche vom KJPD organisiert wurden, Eintrittsgepsräche auf Station und Austrittsgespräche.
        \item Schreiben von Verlaufseinträgen im eingesetzten Kliniktool (Polypoint), welche sich aus unterschiedlichen Patienten-Begegnungen auf der Akutstation ergeben haben, unter anderem aus Einzelgesprächen, Gruppen-Therapien, KJPD Infoveranstaltungen, Bewegungstherapien und Spaziergängen.
        \item Co-Leitung und einmalige Leitung der wöchentlichen Gruppentherapie.
        \item Führen von Einzelgesprächen ausgewählter Bezugspatienten.
        \item Co-Leitung der Psychoedukation, die vier Wochen andauerte.
        \item Teilnahme und Mitgestaltung der wöchentlichen Literaturgruppe.
        \item Teilnahme und Mitgestaltung der Morgenrunde, die jeden Tag stattfand.
        \item Einmalige Erstbetreuung eines ISO-Patienten (Alkoholtestung, Urinal, Zigarettenpause).
        \item Teilnahme an internen Weiterbildungsangeboten (\textit{AMDP Seminar, Elternschaft mit Borderline})
        \item Begleiten von Patienten wie den täglichen Spaziergängen, Begleitung zum Kiosk, Begleitung zu den Therapien und Begleitung für Besorgungen zu den Patienten nach Hause.
        \item Zudem aktive Teilnahme am Therapieprogramm, um einen besseren Einblick in den Patientenalltag zu erlangen: Musiktherapie, Ergotherapie, Bewegungstherapie, Entspannungsübungen und Weiterbildungsseminare.
    \end{itemize}
\subsection{Persönliche Auseinandersetzung mit dem Praktikum} \label{sec:Auseinandersetzung}

%----------------------------------------------
\subsubsection{Auswertung ausgewählter Funktionen}
In diesem Abschnitt werde ich genauer auf einzelne ausgeführte Funktionen hinsichtlich meiner an der Klinik gemachten persönlichen Erfahrungen eingehen. Was habe ich durch die Ausübung dieser Aufgaben gelernt? Konnte ich auf dem Wissen meiner Ausbildung aufbauen? Welche Ressourcen habe ich dabei eingesetzt und wo konnte ich Wissenslücken bei mir feststellen? 

Genauer unter die Lupe nehmen werde ich den Patientenkontakt im allgemeinen, die Psychoedukation, die Gruppentherapie und die Einzelgespräche. Aus meiner Sicht konnte ich mich bei diesen Tätigkeiten gut einbringen und hinsichtlich meiner Stärken und Schwächen am meisten profitieren. 

\paragraph{Allgemeiner Patientenkontakt auf der Station}
Bevor ich mein Praktikum startete, machte ich mir diverse Gedanken über den Patientenkontakt. Diesem würde ich mich in den mir bevorstehenden drei Monaten wohl oder übel aussetzen müssen. Bis dato wusste ich nicht, wie ich mich im direkten Kontakt verhalten würde. Mein Wissen über affektive Störungen und Persönlichkeitsstörungen war vorwiegend theoretischer Natur. Das was wir in Form unseres Studiums aus Büchern und Vorlesungsunterlagen gelernt haben. Natürlich gab es da ein paar direkte Begegnungen mit Menschen, die das eine oder andere "Burnout" hinter sich hatten. Nur, jetzt ging es um einen Aufenthalt in einer Akutklinik und nicht um mir bekannte Personen. Einerseits war ich total nervös und unsicher. Würde ich das packen? Würde ich es aushalten mit Menschen in den Kontakt zu treten, die an einer starken psychischen Krankheit leiden? Würde ich eine Station voller Irrer antreffen, die den ganzen Tag in einer Zwangsjacke über die Station gehen? Bitte entschuldigen sie hier meine Wortwahl. Mir ist durchaus bewusst, dass diese Bezeichnung sehr abwertend und unreflektiert ist. Aber es geht hier um meine persönliche Reflexion und die möchte ich so ehrlich wie möglich gestalten. Dieses Bild war halt dasjenige, welches vor meinem Praktikum in meinem Kopf war. Ich schäme mich im Nachhinein dafür. In anderen Worten, ich hatte Angst vor diesem Praktikum und erwartete ähnliche Verhältnisse wie im Spielfilm "Einer flog über das Kuckucksnest", mit Jack Nicholson. Ich, als ehemaliger Software-Ingenieur, bei dem sich die menschliche Interaktion bei einem Kundenkontakt auf dem Höhepunkt befunden hat und ansonsten die meiste Zeit in einen farbigen Monitor blickte, ging in direktem Weg in die Konfrontation mit meinen innersten Ängsten. Andererseits war ich auch extrem gespannt, was mich erwarten würde. Was für Persönlichkeiten treffe ich da an? Wie fest würde ich mit den Patienten interagieren dürfen? Werde ich überhaupt etwas im direkten Kontakt machen dürfen? Sind da Menschen, bei denen die Krankheit klar ersichtlich ist oder würde ich ihre Störungen ohne Diagnose nur schwer erkennen? Würde ich einen Unterschied zwischen einer leichten depressiven Episode und einer schweren Episode erkennen? Würden sich die ICD-10 Diagnosen ersichtlich zeigen oder werden diese Primär für die Abrechnung mit den Kassen vorgenommen? Dies waren nun also meine Gedanken und mit etwas zittrigem Schritt trat ich mein Praktikum im stationären Akutbetrieb an.

An meinem ersten Tag wurde meine Vorstellung über das Aussehen der Station komplett revidiert. Da befanden sich keine weiss getünchten Wände und auf dem Gang schleppten sich auch keine gebückt gehenden Patienten mit Infusionsständer den Wänden entlang. Vielmehr sah es etwas aus wie auf einem - zugegeben in die Jahre gekommenem - Hotelflur, mit Ausnahme der Stationstür, die war zu diesem Zeitpunkt geschlossen. Aber weit und breit kein Mensch zu sehen (es war auch Frühstückszeit und die Patienten waren im Aufenthaltsraum). Im Stationszimmer dann die nächste Überraschung: Keine griesgrämig dreinblickenden Pflegeschwestern in weissen Kitteln. Nein, alles aufgestellte und alltäglich angezogene Pflegerinnen. Ich wurde herzlich empfangen und als erstes mit der Kaffeemaschine vertraut gemacht. Hm, bis dahin war alles ziemlich normal. Als ich dann mit dem Oberarzt bekannt gemacht wurde und er meinte, er sei der Walter, viel mein inneres Bild der überheblichen Ärzte und rigiden Strukturen innerlich zusammen. Doch bis zu jenem Zeitpunkt hatte ich noch keinen einzigen Patienten gesehen. Das würde auch noch etwas dauern, denn erst war der interdisziplinäre Rapport angesagt. Da wurde zwar über die Patienten gesprochen, aber ein Bild konnte ich mir immer noch nicht so richtig machen. Endlich ging es in Richtung Patient, ich durfte an der Oberarztvisite teilnehmen. 

Dies war nun die erste Annäherung an die stationären Patienten. Gut geschützt hinter dem Oberarzt, der Psychologin und der Pflege harrte ich nun den Dingen, die da kommen mögen. Als erstes sah ich einen Patienten, der an einer mittelgradigen Depression litt und zudem eine Persönlichkeitsakzentuierung im Bereich emotional instabile Persönlichkeitsstörung diagnostiziert bekommen hatte. Mir begegnete ein sympathisch wirkender Mann im mittleren Alter. Vielleicht etwas bedrückt. Aber niemals das, was ich erwartet hätte.  

In den nächsten Tagen nahm ich an vielen Hospitationen teil. Darunter waren Einzelgespräche und Gespräche im Kernteam, welches aus einer Psychologin und einer Pflegeperson zusammensetzte. Mit jeder Person, die ich kennenlernen durfte, wich mein voreingenommenes Bild und machte Platz für ein humaneres, aufgeschlosseners. 

Mein erster direkte Kontakt, an den ich mich so richtig erinnern kann, war dann am dritten Tag. Meine Praktikumsbegleitung war nicht anwesend. Übergab mir jedoch am Vortag eines ihrer Einzelgespräche. Ich war ziemlich nervös. Es handelte sich um eine Patientin, die bereits sehr klinikerfahren war und zum ca. 90igsten Eintritt auf die Station kam. In ihrer Akte stand die Diagnose Emotional instabile Persönlichkeitsstörung vom Typ Borderline. Vor dem Gesprächszimmer traf ich auf eine Frau etwas über 50 Jahre alt. Ihr Gesicht war mit Falten durchfurcht und sie hatte kurze, zum Bürstenschnitt aufgestellte Haare. Sie wirkte sehr freundlich. Im Gespräch wollte ich als Erstes wissen, was sie dazu bewogen hat stationär zu uns zu kommen. Die Frau gab breitwillig Auskunft. Sie erzählte mir soviel aus ihrem Leben und genoss es sichtlich, dass ihr wieder einmal jemand mit voller Aufmerksamkeit zuhörte. Es entwickelte sich ein gegenseitiges Gespräch, in dem sie auch mich über meine Funktion als Praktikant und meine Ausbildung befragte. Die Stunde, die ich mir dafür Zeit genommen habe, verging wie im Flug und ich beendete mein erstes Gespräch über den guten Verlauf sichtlich erleichtert.

Weitere Gespräche folgten. Mit der Zeit entwickelte ich eine gewisse Übung und legte meine Unsicherheit nahezu ab. Am Ende meines Praktikums gelang es mir immer mehr, egal an welchem Ort, Patienten in ein Gespräch zu verwickeln und ihre Stimmungslage zu erfragen. Teilweise machten wir gegenseitig Scherze auf dem Gang. Die Patienten kamen mit der Zeit immer mehr direkt auf mich zu, da sie in mir einen vollwertigen Gesprächspartner sahen. Sie vertrauten mir Dinge an, die sie nicht mal der direkten Bezugsperson anvertrauten (oder wenigstens sagten sie mir das). Am Ende meines Praktikums fühlte ich mich auf der Station so wohl, dass es mir wie eine zweite Familie vorgekommen ist. Natürlich waren da auch Patienten, bei denen ich mich in acht nehmen musste. Bei einem paranoid schizophrenen Herrn bewegten sich alle vom Team auf sehr dünnem Eis, da er sehr schnell dekompensierte und verbal aggressiv wurde.

Rückblickend schäme ich mich für meine Vorurteile. Kein einziges davon hat sich in der Praxis bewahrheitet. Ich lernte Menschen kennen, die ich in mein Herz geschlossen habe. Ich habe durch das Praktikum die Erfahrung machen dürfen, dass ich den Kontakt zu Patienten als bereichernd wahrnehme. Dementsprechend fiel es mir schwer, am Ende meiner Praktikumszeit wieder zu gehen. Nicht nur mein Abschied vom Team viel mir schwer, auch der Abschied von den Patienten, die nach der Akutphase entweder nach Hause, oder in die Tagesklinik übergetreten sind, vielen mir nicht immer leicht. Aus den Begegnungen habe ich gelernt, dass eine empathische und zugewandte Haltung eine positive Wirkung auf das Gegenüber hat. Diese Erfahrungslücke durfte ich durch den Sprung in die Praxis schliessen, denn so etwas lässt sich im Schlungsraum nicht erlernen. 

\paragraph{Gruppentherapie}
Neben diversen Therapieformen wie Musiktherapie, Bewegungstherapie, Ergotherapie, Einzelgespräche und andere Therapieformen, wurde einmal pro Woche eine Gruppentherapie durchgeführt. Diese wurde von der stationsunabhängigen Fachpsychologin der Klinik durchgeführt, welche von den Assistenzpsychologinnen abwechselnd in der Co-Leitung unterstütz wurde. 

Während meinem Aufenthalt nahm ich jede Woche in der Co-Leitung dieser Gruppentherapie teil. Die meiste Zeit als einziger Co-Leiter, da unsere Psychologinnen unter einer dauernden Unterbesetzung auf der Station litten und deshalb oft nicht in der Gruppe anwesend waren. 

Die Gruppe kam nicht bei allen Teilnehmer gut an. Sowohl die Co-Leiterinnen waren nicht alle über diese Form der Therapie erfreut und versuchten auch deshalb fern zu bleiben. Bei den Patienten stiess diese Form der Therapie oft auf Wiederstände. Einige äusserten, dass sie lieber Einzelgespräche führten. Andere wiederum gaben an, besseres zu tun zu haben, als sich die Sorgen und Nöte anderer Patienten anzuhören. Weiter gab es noch diejenigen, die schlicht und einfach zu wenig gesund waren, um sich um 10 Uhr morgens in einen Gruppenraum ausserhalb der Station zu begeben.

Meine persönliche Meinung über die Gruppentherapie ist folgende: Ich bin ein grosser Fan von Gruppen. Ich gehe davon aus, dass die meisten pyschischen Probleme aus der Interaktion und den Beziehungen mit anderen Menschen entstanden sind. Dadurch ist die Gruppentherapie eine Methode, diese ungünstigen Interaktionen und Beziehungen korrektiv zu beeinflussen. In einer Gruppe können Erfahrungen in einem geschützten Rahmen gemacht werden.

Das Ziel dieser Gruppe war es, die Patienten dazu anzuregen, ihre Gefühle genau zu beschreiben und differenziert zu benennen. Daraus entstanden Themen, über die in der Gruppe diskutiert wurden. In privatem Ramen nahm ich während dem Studium während etwa eineinhalb Jahren wöchentlich an einer Gruppensitzung von 7 Personen teil. Mich faszinieren die Abläufe und die Möglichkeiten einer Gruppe. Ein grosser Fan bin ich natürlich von Irvin D. Yalom, einem amerikanischen Psychoanalytiker und Psychotherapeut, der mein Verständnis von Gruppentherapie nachhaltig prägte \cite{Yalom2010}.

In der Funktion als Co-Leitung durfte ich mich nach eigenem Gutdünken selber einbringen. Immer dazu gehörte ein persönliches Statement am Schluss, welches meine Sicht auf den Prozess widerspiegeln sollte. Natürlich war es auch meine Aufgabe, die Verlaufseinträge der einzelnen Patienten vorzunehmen. Dies stellte sich als die grösste Herausforderung dar, da unter Umständen bis zu 9 Patienten anwesend waren und ich mir deren Prozess während der Stunde zu merken hatte. Natürlich ohne Notizen zu machen, da dies von der leitenden Psychologin so gewünscht wurde und mir als gute Übung diente. Dadurch konnte ich lernen, den Therapieprozess mit dem ständigen Mitschreiben nicht zu stören und meine gleichschwebende Aufmerksamkeit zu schulen. Am Anfang hatte ich grosse Mühe, mir die wichtigen Prozesse zu merken. Mit der Zeit konnte ich mir anhand der Vorstellung der einzelnen Gesichter der Patienten, die wichtigen Statements und Prozesse in Erinnerung rufen. 

Eines Morgens kam die Gruppenpsychologin auf die Station und fragte mich, ob ich die Gruppe selber leiten möchte. Nach etwas Bedenkzeit willigte ich ein und machte meine erste Erfahrung in diesem Bereich. Ich versuchte wie gewohnt auf die aktuellen Gefühle der Patienten Bezug zu nehmen. Ich hatte Glück, denn es waren nur 5 Patienten anwesend und diese waren sehr aktiv dabei. Die leitende Psychologin griff dabei nur sehr wenig in den Prozess ein und half mir wenn nötig unterstützend. Ich habe diese Erfahrung als sehr bereichernd empfunden. Sowohl die Leitung, als auch die Interaktion mit den Patienten. Ebenso die gemeinsame Leitung mit der Psychologin war von Respekt und Vertrauen geprägt. Durch den Prozess, über die Gefühle in eine Diskussion einzusteigen, konnte ich meiner erste Gruppe leiten. Dennoch merkte ich, dass ich damit nicht wirklich zufrieden war. Irgendetwas fehlte. Ich hatte das Gefühl, dass die Gruppe noch nicht am Ende ihrer Möglichkeiten war. Nur wusste ich noch nicht genau was mir in dieser Gruppe fehlte. 

Kurz darauf beschloss das Team auf A2 neue Broschüren über die Therapieangebote zu erstellen. Diese sollen dazu dienen, den Patienten zusätzliche Informationen zum Therapieangebot zu vermitteln. Unter den Aufgaben befand sich die Erstellung eines Flyers für die Gruppentherapie. Die leitende Psychologin bat mich, mit ihr zusammen ein paar Infos zusammenzutragen, die sie dann in den Flyer aufnehmen wolle. Zusammen diskutierten wir, was da alles rein soll. Ich gab an, dass ich es neben den Angaben über die Arbeit mit Gefühlen wichtig finden würde, die Patienten darüber zu informieren, dass die Gruppe ein gutes Instrument sei, korrigierende Erfahrungen in der Interaktion mit Mitmenschen zu machen. Zudem besprachen wir das Vorgehen von \citeA{Yalom2005}, welches er in seinem Buch "Im Hier und Jetzt" beschrieben hat. Ich bekundete mein Interesse an seinem Vorgehen und wollte ihre Meinung dazu hören. Die Psychologin war etwas zurückhaltend, da sie nachweislich einen etwas anderen Ansatz vertrat. Sehr zu meiner Freude liess sie an meiner letzten Gruppenteilnahme einen Teil von diesem Ansatz in die Gruppe einfliessen. Es lief hervorragend. Ob dies an der Grösse von drei Personen lag oder an der aktiven Teilnahme der Patienten, die Stunde entpuppte sich als ein Highlight in meinem Praktikum. Sogar die leitende Psychologin gab am Schluss an, dass diese Stunde eine Sternstunde in ihrer Geschichte von Gruppentherapien gewesen sei. Diese Rückmeldung hat mich sehr gefreut. Konnte ich doch mit meiner Anregung wenigstens etwas im Rahmen der Gruppentherapie bewirken. Mein inneres Gefühl, mich in Richtung personenzentrierter Ansatz weiter zu bilden wurde dadurch weiter gestärkt. Die Erfahrung in der Gruppentherapie gilt für mich als eine der reichhaltigsten im Rahmen meines Praktikums.

\paragraph{Psychoedukation}
Die Psychoedukation zählt zu den Angeboten an der Klinik, die auf allen Stationen in regelmässigen Abständen durchgeführt wird. Dabei werden die Themen entsprechend den einzelnen Schwerpunkten der jeweiligen Stationen angepasst. Zum Beispiel wurde auf der Station A2 eine Vertiefung der affektiven Symptomen vorgenommen. Die Psychoedukation umfasste sieben Einheiten, die über vier Wochen regelmässig durchgeführt wurden. Diese Kurse werden in der Regel von der zuständigen Aktiverungstherapeutin geleitet und folgen einem mehr oder weniger standardisierten Ablauf. Themen in diesen Kursen sind unter anderem das Psychosoziale-Modell, Frühwarnzeichen, unterschiedliche Symptome, ungünstiges Verhalten und Pharmakotherapie.

Meine Aufgabe bestand in der Co-Leitung dieses Kurses, indem ich die Materialien vorbereitete, für jede Kurseinheit ein Handout zusammenstellte, während dem Kurs die Rückmeldungen der Teilnehmer auf einem Flipchart notierte und diese Später in das nächste Handout einpflegte. Zudem übernahm ich Teile der Theorie und förderte die aktive Teilnahme der Patienten, indem ich nachfragte und einzelne Patienten direkt zu ihrer Meinung aufforderte. 

In dieser Aufgabe konnte ich auf die Erfahrung meines Psychologiestudiums zurückgreifen. Durch unzählige Gruppenarbeiten und Präsentationen bin ich es mir gewohnt vor Leuten zu referieren und interaktiv zusammenzuarbeiten. Die vorgestellten Modelle waren mir allesamt sehr gut bekannt und dementsprechend konnte ich mich als Co-Leiter in der Vermittlung der Theorie aktiv einbringen. Mir gefiel diese Rolle sehr gut. Ich hatte das Gefühl, mich dabei entfalten zu können. Es freute mich, konnte ich dabei auf die unzähligen Lernerfahrungen des Studiums zurückgreifen und quasi aus dem Vollen schöpfen. Die Rückmeldung der Leitung war dementsprechend sehr positiv. Ebenso die Rückmeldung der Patienten. Es schien, als ob sie von einem solchen Format der Therapie enorm profitieren konnten, indem sie die Möglichkeit der aktiven Mitgestaltung hatten. Sie schätzten es wohl auch, dass sie zu Wort kamen und dabei eine Expertenrolle bezüglich ihrer eigenen Krankheit zugestanden bekommen haben. Es scheint den Patienten wichtig zu sein, nicht nur passiv auf ein Urteil von einem Arzt oder Psychologen angewiesen zu sein, sondern sich viel mehr aktiv und mit mehr Kompetenz einbringen zu können.

Diesen Punkt finde ich wichtig, da auch ich mich im Verlauf des Praktikums oftmals wunderte, wie wenig den Patienten bezüglich ihrer eigenen Krankheit zugetraut wurde. Mein Verständnis scheint da mehr vom humanistischen Gedankengut geprägt zu sein, in dem ich davon ausgehe, dass die Patienten mehr über ihre Krankheit wissen, als jeder Aussenstehende es je könnte. Deshalb würde ich es sehr begrüssen, würde eine Klinik in ihrem Therapieangebot stärker auf die Bedürfnisse der Patienten eingehen. Ich werde auf diesen Punkt weiter unten näher eingehen (\prettyref{sec:Höhepunkte}).

\paragraph{Einzelgespräche}
Auf diesen Punkt möchte ich abschliessend von diesem Kapitel gerne etwas näher eingehen. Wie bereits oben erwähnt, durfte ich bereits an meinem dritten Tag ein Einzelgespräch führen. Dabei fühlte ich mich zugegebenermassen etwas überfordert. Ich nahm jedoch die Herausforderung an und sprang ins kalte Wasser. Im Vorfeld wusste ich nicht viel über die mir zugeteilte Patientin. Nach der anfänglichen Panik, ausgelöst durch die Vorstellung alleine eine Patientin zu empfangen, löste sich meine Erstarrung und ich konnte etwas ruhiger an die Aufgabe herantreten. Ich hielt mir vor Augen, dass mir die Patientin von einer erfahrenen Psychologin zugeteilt worden ist. Also schloss ich daraus, dass sie schon wissen musste was sie tat. Zweitens sah ich in dieser Aufgabe eine riesige Chance, als Praktikant etwas auszuprobieren, was ich später vielleicht nicht mehr so einfach konnte. Deshalb wollte ich so wenig wie möglich im Vorfeld über die Patientin wissen. Sie sollte mir im Gespräch soviel sagen dürfen, wie sie es für richtig hielt. 

Dies war mein erster Versuch im Umgang mit Einzelgesprächen. Im obigen Beispiel kam es richtig gut heraus. Die Patientin fasste relativ rasch vertrauen und erzählte mir diejenigen Episoden aus ihrem Leben, die sie mir mitteilen wollte. Damit war mein Ziel erreicht und der Behandlungsauftrag von der Psychologin ebenso. Die Patientin ist in der Klinik bekannt und deshalb war mein Gespräch für die Behandlung nicht unbedingt relevant. Für mich war es jedoch eine tolle Möglichkeit, ein erstes Gespräch an einer psychiatrischen Klinik führen zu können.

Im Verlauf meines Praktikums hatte ich des öfteren die Möglichkeit Einzelgespräche zu führen. Dabei habe ich mich immer wieder auf meinen Praktikantenstatus berufen und die damit verbundene Möglichkeit ausgeschöpft, Erfahrungen zu sammeln. Diese Erfahrungen konnte ich mit meiner zugewiesenen Fachpsychologin alle ein zwei Wochen während eines Coachings besprechen. Durch diese Möglichkeit, die mir die Klinik geboten hat, konnte ich mich weiter für die humanistische Therapie begeistern. Durch mein empathisches Zuhören und dem Wunsch, den Patienten so gut wie mögliche verstehen zu wollen, fassten diese schnell Vertrauen und erzählten mir wichtige Dinge, die für die Behandlung relevant waren. Natürlich hatte ich keinen direkten Behandlungsauftrag. Ich hatte die Narrenfreiheit, die es nur als Praktikant gibt. Diese Freiheit nutzte ich auf alle Fälle und konnte dadurch enorm profitieren.

%----------------------------------------------
\subsubsection{Persönliche Höhepunkte und kritische Punkte} \label{sec:Höhepunkte}
Einige Höhepunkte, die ich innerhalb ausgewählter Tätigkeiten erlebt habe, wurden bereits im vorhergehenden Abschnitt behandelt. Hier gehe ich auf persönliche Höhepunkte und kritische Punkte ein, die mir durch den längeren Aufenthalt in der Klinik mit der Zeit bewusst wurden. 

\paragraph{Veränderter Umgang mit Menschen}
Durch den häufigen Personenkontakt auf der Station, sei es mit dem Pflegepersonal, den Ärzten, den Assistenten oder sonstigem Personal auf der einen Seite, sowohl aber auch durch den direkten Patientenkontakt auf dem Flur oder im Gespräch auf der anderen Seite, veränderte sich mein Umgang mit den Menschen im Allgemeinen. Dies viel mir vor allem ausserhalb der Klinik auf. Im Alltag befinde ich mich immer wieder in Situationen, in denen fremde Menschen auf mich zukommen und etwas von mir wollen. Sei dies, weil sie etwas zu verkaufen haben oder weil sie den Kontakt zu mir suchen. Die Ersteren bereiteten mir nicht allzu grosse Mühe, bei der zweiten Kategorie hatte ich in der Vergangenheit eher meine Hemmungen. Waren dies oft Menschen, die etwas abseits der Norm standen und etwas aus der Menge hervorstachen und auffielen. Ich versuchte immer freundlich zu sein, war aber auch eher zurückhaltend ihnen gegenüber. Durch mein Praktiktum scheint mir der Kontakt zu solchen Personen offener und persönlicher zu gelingen. Ich traue mir mehr zu. Kann viel unbesschwerter und gelöster in eine Interaktion treten. Das früher Unbekannte ist durch mein Praktikum vertrauter geworden. Durch die vielen Menschen, die ich kennenlernen durfte, konnte ich viele bereichernde Erfahrungen machen. Mein Menschenbild hat dadurch eine neue, bis dato unbekannte Facette bekommen. Dieser neue Umgang wirkt sich auch in meinem bekannten Umfeld aus. Es scheint, als ob ich viel ruhiger und gelassener in eine Interaktion treten kann. Eher zugewandt und nicht mehr so sehr mit mir selber beschäftigt. Als ob ich durch den regelmässigen Umgang mit Menschen mehr Selbstvertrauen und ein besseres Selbstbild von mir gewonnen hätte. 

Diese Erkenntnis bestätigt eine Vermutung, die ich schon lange hege: Ich als Person bin auf das Aussen angewiesen. Den Austausch mit Menschen. Die Interaktion mit ihnen. Nicht nur sporadisch, sondern wirklich regelmässig. Ich neige leider dazu, mich in meinen vier Wänden zurückzuziehen und Dinge auszusitzen. Vor allem in etwas schwierigen Zeiten. Dies scheint nicht sonderlich förderlich für mein Wohlbefinden zu sein. Ich merkte dies bereits im Studium. Da tat der Austausch mit Gleichgesinnten sehr gut. Generell der Austausch mit Mitmenschen. Dies mussten nicht mal Mitstudierende sein. Sich einer anderen Person zu öffnen und sich von einer verletzlichen Seite zu zeigen verhalf mir oftmals dazu, die schwierigen Zeiten besser bewältigen zu können. Auch zu hören, dass es Anderen ähnlich ergeht. 

Aus dieser Erfahrung nehme ich mit, dass ich mir für mein zukünftiges Arbeitsfeld eine Stelle wünsche, in der ein reger Austausch mit Menschen stattfindet. Kein Einzelkämpferberuf wie als Softwareingenieur, der ich einmal war oder auch als Student. Schon damals in meinem ursprünglichen Beruf fand ich die interdisziplinäre Zusammenarbeit bereichernd. Zudem bin ich der festen Überzeugung, dass unterschiedliche Blickwinkel zu einem besseren Verständnis beitragen. Ein gutes Team, das zusammenarbeitet. Das wünsche ich mir. Wo das sein wird, steht leider noch in den Sternen. Durch das Praktikum in einem psychiatrischen Betrieb hatte ich einen Einblick in ein Team unterschiedlicher Disziplinen. Leider spielt da die strenge Hierarchie meinem Verständnis von Zusammenarbeit diametral entgegen. Doch dies ist ein anderes Thema und soll weiter unten etwas näher beleuchtet werden (\prettyref{sec:Sonstiges}).

\paragraph{Massenabfertigung vs. individuelle Therapiegestaltung}
Auf einen weiteren aus meiner Sicht kritischen Punkt möchte ich in dieser Arbeit eingehen: Der Umgang mit den Patienten in der Psychiatrie. Durch meine fehlende Erfahrung im Bereich des Gesundheitswesens, insbesondere von stark hierarchischen Systemen, hatte ich zu Beginn meines Praktikums eine nahezu ungetrübte Sicht auf die Psychiatrie (ausgenommen all die einschlägigen Geschichten und Filmen über diesen Bereich). Meine Sicht nach knapp acht Jahren Studium in der Psychologie setzt den Patienten in den Mittelpunkt. Nach meinem Verständnis sollte sich eine Institution um die Bedürfnisse der Patienten anordnen. Das Wohl und der Genesungsprozess sollte meines Erachtens im Zentrum stehen. Mit dieser Einstellung und der Erwartungshaltung, dass dies auch in einer Klinik so ist, die sich um psychisch kranke Personen kümmert, habe ich mein Praktikum begonnen. Nach knapp drei Monaten musste ich meine Erwartungshaltung revidieren. Meine Einstellung ist jedoch geblieben. 

Ich musste schnell lernen, dass eine Klinik eine Firma ist, die primär Geld verdienen muss. Wenn die Betten, bei uns waren es 55 von 60, nicht belegt waren, so machte das Spital Minus. Nachdem dieses Jahr eine neue Tarifordnung eingeführt wurde, veränderte dies nochmals die Einstellung zu den Patienten. Ich würde sagen, dadurch wurde der Patient weiter aus dem Zentrum gerückt. Es geht darum, dass ein Patient je länger er in der Klinik verweilt, weniger Geld für die Klinik einbringt. Zynische Personen würden nun behaupten, dass es nun aus Sicht der Klinik darum geht, Patienten so rasch wie möglich zu stabilisieren und so rasch wie möglich zu entlassen, damit neue Patienten die Betten füllen. Egal wie die Patienten ausserhalb der Klinik zurecht kommen. Besser sie kommen immer wieder, als dass sie für ihre Gesundung länger stationär bleiben würden. Natürlich ist dies eine gewagte Behauptung und trifft nicht generell zu. Dies ist meine persönliche Einschätzung basierend auf meinen Erfahrungen in der kurzen Zeit in der Psychiatrie. 

Zudem ist das Programm im Klinikalltag für die Patienten vorgegeben. Es hat ein Angebot und das Ziel ist es, dass die Patienten dieses Angebot nutzen. Eine individuelle Gestaltung des Therapieangebotes ist dadurch nur beschränkt möglich. Auch wenn sich die Assistierenden Psychologinnen und Ärzte alle Mühe geben. Mein Verständnis von Mensch ist ein gänzlich anderes. Ich bin der Meinung, dass jeder Mensch sehr individuell im Bezug zu seiner Genesung ist. Nicht jedem hilft die Gruppentherapie oder die Ergotherapie. Vielen hilft sie vielleicht, aber nicht allen. Ich bin der Meinung, dass viel stärker auf die einzelnen Patienten eingegangen werden müsste, wenn von einer möglichen Heilung gesprochen wird. Meine Erfahrung war es, dass es primär darum geht, Patienten aufzustellen (stabilisieren) und zu entlassen, denn die Zeit für eine individuelle Betreuung fehlt schlicht und einfach. Sie würde zu viel kosten. 

Im Hinblick auf ein mögliches Tätigkeitsfeld nach dem Studium muss ich mir gut überlegen, ob ich da arbeiten möchte. Kann ich mich mit den Rahmenbedingungen abfinden und versuchen das Beste für mich und den Patienten daraus zu machen, so wie es so mancher Assistent in der Klinik tut. Oder muss ich mir ein anderes Umfeld suchen, eines, wo der Mensch noch eher im Zentrum steht? Diese Überlegungen werde ich mir vor dem August 18 machen müssen. Es wird mir nicht einfach fallen, denn so wie es aktuell im Bereich des psychologischen Arbeitens aussieht, sind die Stellen als Studiumsabgänger nicht allzu dicht gesät.
    
%----------------------------------------------
\subsubsection{Was es sonst noch aus meiner Sicht zu sagen gib} \label{sec:Sonstiges}
In diesem Bereich werde ich auf ausgewählte und mir wichtige Punkte kurz eingehen. Diese sind mir im Verlauf meines Praktikums begegnet und haben mich zum Nachdenken angeregt. Einige davon machen mich wütend, andere wiederum habe ich als Bereicherung aufgenommen. Alle sind für meine Weiterentwicklung wichtig und werden einen Einfluss auf mich für meinen weiteren Weg haben.

\paragraph{Ein Psychologe unter Psychiatern}
Als angehender Psychologe interessierte mich der Umgang zwischen den Ärztinnen und den Psychologen. Wie fand die Zusammenarbeit dieser beiden Berufsgruppen statt. Im Vorfeld hörte ich diverse Geschichten über narzisstische Ärzte und unfähiges psychologisches Personal. 

Ich kann diese Geschichten nicht generell bestätigen. In der psychiatrischen Klinik Breitenau werden Psychologinnen und Psychologen in Ausbildung mit den Assistenzärztinnen und Ärzten auf gleicher Stufe eingestellt. Dies gilt auch für die Entlöhnung. Die Unterscheidung findet sich in der Möglichkeit,  Medikamente auszustellen. Beide Berufsgruppen müssen Tages-, Nacht- und Wochenenddienste leisten. Rein von diesem Punkt gesehen, stehen sich die beiden Gruppen sehr nahe. Doch wie sieht es bei abgeschlossener Ausbildung als Fachpsychologe aus? In der ganzen restlichen Klinik werden in Schaffhausen drei Fachpsychologinnen und Psychologen angestellt. Dies im Gegensatz zu sieben Fachärzte in leitender Position. Bereits aus diesen Zahlen wird ersichtlich, wie unterschiedlich die Berufsgruppen besetzt sind. Nicht nur dass leitende Positionen einen vorwiegend medizinischen Hintergrund haben, auch von der Verteilung in den klinischen Bereichen. Zwei Psychologinnen und Psychologen arbeiten in der Tagesklinik und sind im administrativen Sektor tätig. 

Der Schwerpunkt der Klinik liegt eindeutig auf der pharmazeutischen Behandlung der Patienten, nicht auf der psychotherapeutischen. Dadurch wird mehr medizinisch geschultes Personal benötigt. Als zukünftiger Psychologe auf der Akutstation würden meine Möglichkeiten stark vom zuständigen Oberarzt (es hat aktuell keine Oberärztinnen) und dem leitenden Arzt abhängen. Je nachdem wie offen sie der psychotherapeutischen Behandlung gegenüber eingestellt sind. Ein Psychologe in Ausbildung hat einmal gesagt, dass wir Psychologen in der Klinik Breitenau nichts weiter als amputierte Psychiater wären. Für alles müssten wir den Oberarzt oder den Hintergrunddienst um Erlaubnis fragen. 

Aus meiner Sicht fristet die Psychologie an dieser Klinik ein stiefmütterliches Dasein. Es gibt therapeutische Angebote, nur zählen die im Vergleich zu den Psychopharmaka eindeutig weniger. Ein sich gegenseitig stützendes Konzept fehlt aus meiner Sicht. Dies würde bedeuten, dass nicht der Oberarzt am Ende des Tages über die Behandlung entscheidet, sondern ein gleichberechtigtes Team bestehend aus Psychologinnen, Pflegepersonal und Ärzte sollen Behandlungsvereinbarungen zusammen mit dem Patienten erarbeiten. Jeder Berufszweig hat unterschiedliches Wissen und sollte dies für das Patientenwohl einbringen können. 

Zudem wird das psychologische Wissen viel zu wenig genutzt. Eine Ärztin oder ein Arzt hat in der Regel fünf Jahre Studium und ein Praktikumsjahr hinter sich. Von einer psycholgischen Schulung herzlich wenig. Wir Psychologinnen und Psychologen haben ein Studium in Psychologie, das fünf Jahre andauert und anschliessend immer noch nicht zum Therapeuten befähigt. Die Psychotherapieausbildung kostet viel Geld und dauert lange. Was läuft hier schief? Die Frage besteht weiterhin, soll ich mich in einem solchen Bereich weiterbilden, in dem ich - egal wie weit ich mich noch Bilde - immer zweite Geige spielen werde? Immer den Medizinern untergeordnet bleiben werde? Die Arbeit nicht so ausführen kann, wie wir es im Studium gelernt haben? Werden hier Psychologinnen und Psychologen ausgebildet, die in der Klinik niemand wirklich möchte? Wenn die Psychiatrie mehr medizinisches Personal für die Rekrutieren zur Verfügung hätte, wären Psychologen nicht an der gleichen Position wie sie heute sind? Ist es nicht etwas beschämend, zwecks Arbeitskräftemangel als Notlösung zu dienen, nicht aber wegen den fachlichen Kompetenzen? 

Es bleibt auf alle Fälle ein fahler Geschmack übrig. Auch im Hinblick auf in meinem Fall acht Jahre Ausbildung. Eine Ausbildung, die im Bereich der klinischen Psychologie mit dem Masterabschluss erst richtig beginnt. Wer ausser gut betuchte soll sich das leisten können? Insbesondere hier gilt es den Kopf nicht in den Sand zu stecken und in Form einer positiven kognitiven Umbewertung weiter zu machen. Es lehrt einem fürs Leben. Nur - sie erlauben die zynische Bemerkung - lässt sich davon keine Familie ernähren.  

\paragraph{Stellung und Akzeptanz im Team}
Als Praktikant nahm ich auf der Station A2 eine besondere Stellung ein. Ich war in einem Psychologiepraktikum einerseits einer Psychologin in Ausbildung, andererseits der Aktivierungsleitung der Pflege unterstellt. Meine betreuende Psychologin hatte noch nicht viel Erfahrung im Umgang mit Praktikanten. Sie machte dies zum ersten Mal. Sie holte ihr wissen bei anderen Psychologinnen ein. Gleich zu Beginn war klar, dass die Klinik kein eigentlicher Leitfaden für die Betreuung von Psychologiepraktikantinnen hat. Es gab ein ziemlich in die Jahre gekommener Leitfaden, der besagte, dass die Praktikanntinnen der Aktivierungspflege angegliedert sind. In meinem Fall hatte ich jedoch das Glück, dass meine zuständige Aktivierungstherapeutin unfallbedingt im ersten Monat ausgefallen ist. Somit konnte ich mich an die Fersen meiner Psychologin anheften und den Klinikablauf aus Sicht von ihr miterleben (was aus meiner Sicht eindeutig der sinnvollere Weg darstellt). Ich nahm an Rapporten und Visiten teil, durfte bei den Weiterbildungen dabei sein und war als Hospitant bei den meisten therapeutischen Gesprächen dabei. Durch die unklare Handhabung des Psychologiepraktikums hatte ich freie Hand und durfte mich da einbringen, wo ich mein Interesse bekundete. 

Im zweiten Monat, als ich mich der Aktivierungstherapeutin anschliessen konnte, habe ich gute Erfahrungen aus Sicht der Pflege machen dürfen. Ich hatte einen direkteren Patientenkontakt, da ich mich hautpsächlich auf der Station befunden habe. Dadurch kam ich automatisch in den Austausch zwischen Patienten und Pflege. 

Rückblickend gesprochen bin ich sehr froh, hatte ich die Möglichkeit mich selber zu organisieren. Eine zu enge Einbindung in der Pflege hätte mir nicht zugesagt. Dazu muss noch gesagt werden, dass auch die Aktivierungstherapeutin sehr auf meine Interessen eingegangen ist. Sie liess mich mehr oder weniger machen und förderte mich wo sie konnte. Trotzdem zeigt dieser Umgang den Stellenwert eines Praktikanten auf, wenn dieser aus Sicht der Leitung primär der Pflege angeschlossen ist. Dies führte bei einer Mitpraktikantin auch zu manch hitziger Diskussion. Sie wurde im Gegensatz zu mir von der Pflege sehr fest eingebunden. Sie durfte nahezu keine psychologische Tätigkeiten mitmachen. Stattdessen durfte sie die Patienten füttern und Ausscheidungen aufputzen. 

Ich denke, dass mir hier mein Alter positiv zugespielt hat. Das Team hat mich immer als vollwertige Person und Hilfe akzeptiert. Der Umgang war von gegenseitigem Respekt und Achtung vor der jeweiligen Profession geprägt. Dadurch konnte ich mich zu meinem Wohl voll und ganz entfalten und hatte Einblick in so manch spannende Tätigkeit. Ich übernahm Aufgaben der Pflege, weil sie mich interessierten. Wie zum Beispiel die Überwachung und die Betreuung eines Iso-Patienten. Das Spazierengehen mit einer Gruppe von Patienten. Auf der anderen Seite machte ich Einzelgespräche, nahm an den täglichen Raporten teil und wirkte aktiv in der Co-Leitung mit. Durch die gute Akzeptanz auf der Station kam es mir nie wie ein Praktikum vor. Ich fühlte mich als Teil des Teams und dabei fühlte ich mich sehr wohl.  

\paragraph{Praktikumsleitung}
Ich möchte hiermit noch kurz auf die Praktikumsleitung oder besser gesagt auf die fehlende Praktikumsleitung eingehen. Wie im Kapitel \titleref{sec:Rahmenbedingungen} bereits ersichtlich, ist die Leitung für die Praktikas in der Klinik nicht wirklich geregelt. Ich hatte den Eindruck, dass alle mitreden wollten, aber bei konkreten Fragen und Problemen sich niemand zuständig fühlte. Das liegt aus meiner Sicht daran, das der allgemeine Stellenwert von Praktikanten an der Klinik eher gering ist. Das heisst nicht, dass wir vom Team nicht geachtet wurden. Vielmehr heisst dies, dass es keine gültige Regelung für uns gibt. Es existieren veralteten Richtlinien, in denen steht zum Beispiel, dass eine Psychologiepraktikantin oder Psychologiepraktikant der Aktivierungstherapeutin der Pflege unterstellt ist. Dies hat aber den Nachteil, dass diese eine Praktikantin oder Praktikanten so sehr in die Pflege einbinden kann, dass die oder der viel zu wenig vom eigentlich Psychologischen auf der Station mitbekommt. 




%####################### 
\section{Intervisionsreflexion}\label{sec:Intervision}
\begin{itemize}
    \item Rahmen (Skype, Personen, Vorgehen)
    \item Eigener Fall - Reflecting Team (Pat. Vorstellen, Mühe mit, Rückmeldungen Team)
\end{itemize}

%####################### 
\section{Fazit}\label{sec:Fazit}


Kennengelernte Störungsbilder (Borderline, Depression, Bipolare-Störung, Schizoaffektive-Störung, Manische-Episode, Suchterkrankung, Hebephrene-Schizophrenie)

Wie weiter als Psychologe.



% Literaturverzeichnis
\begin{flushleft}
\nocite{*}
\bibliography{Literaturverzeichnis}{}
\end{flushleft}

\end{document}
