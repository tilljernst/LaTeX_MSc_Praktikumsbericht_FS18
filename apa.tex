\documentclass[jou,apacite]{apa6}
\usepackage[utf8]{inputenc} % für die Umlaute

% Für Verweise innerhalb Doc (siehe https://strobelstefan.org/?p=145)
\usepackage{csquotes}
\usepackage{prettyref}
\usepackage{titleref}
%%% Für Abschnitte %%%
\newrefformat{sec}{siehe Abschnitt~\ref{#1} \enquote{\titleref{#1}} \ auf Seite \pageref{#1}}
%%% Für Abbildungen %%%
\newrefformat{fig}{siehe Abb.~\ref{#1} \enquote{\titleref{#1}} \ auf Seite \pageref{#1}}
%%% Für Tabellen %%%
\newrefformat{tab}{siehe Tab.~\ref{#1} \enquote{\titleref{#1}} \ auf Seite \pageref{#1}}


\title{Praktikumsbericht ER4 - Psychiatriezentrum Breitenau}
\shorttitle{Shorttitle}

\author{Stud. MSc Till J. Ernst}
\affiliation{Applied Psychology ZHAW}

\abstract{Im folgenden Dokument soll aus persönlicher ...}

%\rightheader{APA style}
%\leftheader{Author One}

\begin{document}
\maketitle    
   
%##############################################
\subsection{Einleitung} \label{sec:Einleitung}
TBD

%##############################################
\subsection{Rahmenbedingungen und Fakten (TBD: 1 bis 2 Seiten)} \label{sec:Rahmenbedingungen}

%----------------------------------------------
\subsubsection{Ort, Dauer und Entlöhnung} 
Das Praktikum im Rahmen der Ausbildung an der ZHAW zum Psychologen führte ich an der \textit{Klinik für Psychiatrie und Psychotherapie} aus, welche als psychiatrischen Dienst der Spitäler Schaffhausen angegliedert ist. Die Klinik hat vom Kanton Schaffhausen einen Versorgungsauftrag zur stationären, teilstationären und ambulanten Behandlung und Betreuung erwachsener psychisch Kranker. Die Klinik verfügt über 60 Betten in den Akutpsychiatrie und Rehabilitation. Zudem sind 12 Plätze in der psychiatrischen Tagesklinik vorhanden. Das Gebiet für Versorgung umfasst einerseits den gesamten Kanton Schaffhausen, sowie die angrenzenden Gebiete. Die Klinik arbeitet eng mit weiteren Kliniken des Kantons Zürich zusammen und nimmt bei Bedarf (wenn die angeschlossenen Kliniken keine Betten mehr haben) Patienten aus Zürich auf.

Das Praktikum dauerte vom 3.1.2018 bis und mit 30.3.2018 und umfasste insgesamt 433,5 Stunden zu 8,5 Stunden pro Tag bei einer Anstellung von 100\% (gemäss Vorgaben sind im Masterstudiengang 300h vorgesehen). 

Die Entlöhnung bei 100\% umfasste pro Monat CHF 1000.- (brutto) und wurde per Ende Monat auf das Lohnkonto überwiesen. Zusätzlich wurde ein Teil des 13ten Monatslohns Ende März überwiesen. Das Praktikum entsprach einer herkömmliche Anstellung am Spital. Aus diesem Grund wurden mit in dieser Zeit 5,5 Tage Ferien zur Verfügung gestellt. Zudem wurde ich an den in diesem Semester fälligen Unterrichtstagen an der ZHAW freigestellt. 

Für diesen drei Monaten wurde ich fest auf der Akutstation für affektive Störungen und Persönlichkeitsstörungen eingeplant. Diese Station hatte die Bezeichnung A2 und besitzt eine Kapazität von rund 20 Betten. Zudem verfügt diese Station über einen Isolationsraum, der bei Bedarf eingesetzt werden kann. Die restlichen Zimmer sind in Zwei- und Dreibettzimmer aufgeteilt, wobei die Kapazität im Notfall mit sogenannten Überbetten erhöht werden kann. 

In den drei Monaten hatte ich die Möglichkeit neben der Station A2 weitere Stationen in Form einer Tageshospitation zu besuchen. Ich nahm von diesem Angebot gebrauch und verbrachte je einen Tag auf der Station für psychotische Störungen und in der Tagesklinik. 

%----------------------------------------------
\subsubsection{Praktikumsleitung (Person / Funktion)}
Im Psychiatriezentrum gab es bis zu meinem Austritt kein Einzelperson, die für die Praktikumsleitung verantwortlich war. Die Leitung wurde anhand unterschiedlicher Teilbereichen unter verschiedenen Personen aufgeteilt: Für die zeitliche Planung und Auswahl der Praktikanten und Praktikantinnen zeichnete sich Frau Sybille Echte-Schnauber verantwortlich, Fachpsychologin Psychologischer Dienst Psychiatrie. Sie war Ansprechperson für die Bewerbung und war für die Auswahl der Psychologiepraktikantinnen und Praktikanten zuständig. Für den Inhalt des Praktikums zeichnete sich Herr Dr. biol. hum. Bernd Lehle verantwortlich. Aktuell leitender Psychologe der Tagesklinik. 

Diese beiden oben erwähnten Personen waren für die Auswahl und den groben Ablauf des Praktikums zuständig. Da alle Praktikantinnen und Praktikanten einer bestimmten Station angeschlossen sind, werden sie als angehende Psychologinnen und Psychologen automatisch dem zuständigen Oberarzt unterstellt. In meinem Fall war dies Herr Dr. med. Walter Brogiolo, Therapeutischer Bereichsleiter Akutpsychiatrie und Therapeutischer Leiter A2. Wie das in Kliniken Usus ist, möchte ich hier die oberste Instanz nicht unerwähnt lassen, welche durch den Chefarzt Herr PD Dr. med. Bernd Krämer abgedeckt wurde. Ich erwähne ihn aus dem einfachen Grund, da er bei Unstimmigkeiten im Verlauf eines Praktikums, welche nicht vom zuständigen Oberarzt, noch von der Praktikumsleitung gelöst werden kann, intervenierend einzugreifen vermag. Zudem konnte ich Herrn Krämer als an den Prkatikantinnen und Praktikanten interessierten Arzt kennenlernen, der auch gerne Anregungen und Tipps zum individuellen Praktikum abgegeben hat.

%----------------------------------------------
\subsubsection{Angaben über Betreuung, Super- und Intervision}
Die eigentliche Praktikumsanleitung wurde von Frau M.Sc. Martina Piraccini übernommen. Psychologin in Ausbildung auf der Akutstation A2 für affektive Störungen, auf welcher ich in den drei Monaten hauptsächlich angeschlossen war. Sie hatte die Funktion der ersten Ansprechperson und übernahm die Koordination der einzelnen Praktikumstätigkeiten. Diese Aufgabe hat sie mit der zuständigen Aktivierungstherapeutin, Frau Birgitta Busin, wahrgenommen, welche für meine Einbettung in die Pflege verantwortlich war. 

Einerseits führte das Pflegeteam monatlich eine Supervision durch. Zudem wurde monatlich eine Fallsupervision mit dem gesamte Team der Akutstation A2 (von der Pflege zum Oberarzt, über die Psychologinnen und Assistenzärztin der Station) durchgeführt. Diese Supervisionen wurden beide von einer externen Fachperson, Herr Thomas Disler, Eidgenössisch anerkannter Psychotherapeut und Diplomsupervisor, durchgeführt. An beiden Veranstaltungen durfte ich als Praktikant teilnehmen. 

Zudem wurde bei Bedarf ein spezifisch Coaching / Supervision für Praktikanten bei Frau Ronit Rüttimann, Fachpsychologin psychologischer Dienst, am Haus angeboten. Dieses Coaching nahm ich zu Beginn wöchentlich, danach etwas jede zweite Woche wahr. Dieses Coaching diente dazu, Fragen und Probleme im Zusammenhang mit dem Praktikum zu besprechen.

Durch die ZHAW wurde eine durch Studierenden geführte Intervision vorgeschrieben. Diese Intervision wurde von drei Mistudierenden und mir alle paar Wochen durchgeführt. Darin haben wir unsere individuellen Anliegen und Fälle gegenseitig vorgestellt und besprochen (\prettyref{sec:Intervision}). 

%----------------------------------------------
\subsubsection{Auflistung aller ausgeführter Funktionen}
Im Folgenden werde ich alle Funktionen auflisten, die ich in den drei Monaten durchführen konnte. In einem ersten Teil werde ich nur knapp auf diese eingehen und im zweiten Teil auf ausgewählte näher und ausführlicher eingehen.
\begin{itemize}
        \item Diverse Hospitationen unter anderem bei Einzelgesprächen mit Patienten, Oberarzt-Visiten, Chefarzt-Visiten, Morgen-Rapporte, Somatik-Rapporte, Interdisziplinären-Rapporten, Elternrunden, welche vom KJPD organisiert wurden, Eintrittsgepsräche auf Station und Austrittsgespräche.
        \item Schreiben von Verlaufseinträgen im eingesetzten Kliniktool (Polypoint), welche sich aus unterschiedlichen Patienten-Begegnungen auf der Akutstation ergeben haben, unter anderem aus Einzelgesprächen, Gruppen-Therapien, KJPD Infoveranstaltungen, Bewegungstherapien und Spaziergängen.
        \item Co-Leitung und einmalige Leitung der wöchentlichen Gruppentherapie.
        \item Führen von Einzelgesprächen ausgewählter Bezugspatienten.
        \item Co-Leitung der Psychoedukation, die vier Wochen andauerte.
        \item Teilnahme und Mitgestaltung der wöchentlichen Literaturgruppe.
        \item Teilnahme und Mitgestaltung der Morgenrunde, die jeden Tag stattfand.
        \item Einmalige Erstbetreuung eines ISO-Patienten (Alkoholtestung, Urinal, Zigarettenpause).
        \item Teilnahme an internen Weiterbildungsangeboten (\textit{AMDP Seminar, Elternschaft mit Borderline})
        \item Begleiten von Patienten wie den täglichen Spaziergängen, Begleitung zum Kiosk, Begleitung zu den Therapien und Begleitung für Besorgungen zu den Patienten nach Hause.
        \item Zudem aktive Teilnahme am Therapieprogramm, um einen besseren Einblick in den Patientenalltag zu erlangen: Musiktherapie, Ergotherapie, Bewegungstherapie, Entspannungsübungen und Weiterbildungsseminare.
    \end{itemize}

%##############################################
\subsection{Persönliche Auseinandersetzung mit dem Praktikum (TBD: 5 bis 7 Seiten)} \label{sec:Auseinandersetzung}

%----------------------------------------------
\subsubsection{Auswertung ausgewählter Funktionen}
In diesem Abschnitt werde ich genauer auf einzelne ausgeführte Funktionen hinsichtlich meiner an der Klinik gemachten persönlichen Erfahrungen eingehen. Was habe ich durch die Ausübung dieser Funktion gelernt? Konnte ich auf dem Wissen meiner Ausbildung aufbauen? Welche Ressourcen habe ich dabei eingesetzt und wo konnte ich Wissenslücken bei mir feststellen? 

Genauer unter die Lupe nehmen werde ich den Patientenkontakt im allgemeinen, die Psychoedukation, die Gruppentherapie und die Einzelgespräche. Aus meiner Sicht konnte ich mich bei diesen Tätigkeiten gut einbringen und hinsichtlich meiner Stärken und Schwächen am meisten profitieren. 

\paragraph{Allgemeiner Patientenkontakt auf der Station}
Bevor ich mein Praktikum startete, machte ich mir diverse Gedanken über den Patientenkontakt. Diesem würde ich mich in den kommenden drei Monaten wohl oder übel aussetzen müssen. Bis dato wusste ich nicht wie ich mich im direkten Kontakt verhalten würde. Mein Wissen über affektive Störungen und Persönlichkeitsstörungen war vorwiegend theoretischer Natur. Das was wir in Form unseres Studiums aus Büchern und Vorlesungsunterlagen gelernt haben. Natürlich gab es da ein paar direkte Begegnungen mit Menschen, die das eine oder andere "Burnout" hinter sich hatten. Nur, jetzt ging es um einen Aufenthalt in einer Akutklinik. Einerseits war ich total nervös und unsicher. Würde ich das packen? Würde ich es aushalten mit Menschen in den Kontakt treten, die an einer starken psychischen Krankheit leiden? Würde ich eine Station voller Irrer antreffen (bitte entschuldigen Sie hier meine Wortwahl. Mir ist durchaus bewusst, dass diese Bezeichnung sehr unreflektiert ist. Aber es geht hier um eine persönliche Reflektion und die möchte ich so ehrlich wie möglich gestalten und das ist halt das, was sich in meinem Kopf abspielte), die den ganzen Tag in einer Zwangsjacke über die Station gehen? In anderen Worten, ich hatte etwas Angst vor diesem Praktikum und erwartete ähnliche Verhältnisse wie im Spielfilm "Einer flog über das Kuckucksnest", mit Jack Nicholson. Ich, als ehemaliger Software-Ingenieur, bei dem sich die menschliche Interaktion bei einem Kundenkontakt auf dem Höhepunkt befand und ansonsten die meiste Zeit in einen farbigen Monitor blickte, ging in direktem Weg in die Konfrontation mit meinen innersten Ängsten. Andererseits war ich auch extrem gespannt, was mich erwarten würde. Was für Persönlichkeiten treffe ich da an? Wie fest würde ich mit den Patienten interagieren dürfen? Werde ich überhaupt etwas im direkten Kontakt machen dürfen? Sind da Menschen, bei denen die Krankheit klar ersichtlich ist oder würde ich ihre Störungen ausserhalb der Station niemals erkennen? Würde ich einen Unterschied zwischen einer leichten depressiven Episode und einer schweren Episode erkennen? Würden sich die ICD-10 Diagnosen ersichtlich zeigen oder werden diese Primär für die Abrechnung mit den Kassen vorgenommen? Dies waren nun also meine Gedanken und mit etwas zittrigem Schritt trat ich mein Praktikum im stationären Akutbetrieb an.

An meinem ersten Tag wurde meine Vorstellung über das Aussehen der Station komplett revidiert. Da befanden sich ja gar keine weiss getünchten Wände und auf dem Gang schleppten sich auch keine gebückte Patienten mit Infusionsständer umher. Vielmehr sah es etwas aus wie auf einem - zugegeben in die Jahre gekommener - Hotelflur, mit Ausnahme der Stationstür, die war zu diesem Zeitpunkt geschlossen. Aber weit und breit kein Mensch zu sehen (es war auch Frühstückszeit und die Patienten waren im Aufenthaltsraum). Im Stationszimmer dann die nächste Überraschung: Keine griesgrämige Pflegeschwestern in weissen Kitteln. Nein, alles aufgestellte und individuell angezogene Pflegerinnen. Ich wurde herzlich empfangen und wurde als erstes mit der Kaffemaschine vertraut gemacht. Hm, bis dahin war alles ziemlich normal. Als ich dann mit dem Oberarzt bekannt gemacht wurde und er meinte, er sei der Walter, viel mein inneres Bild der überheblichen und straffen Strukturen innerlich zusammen. Doch bis zu jenem Zeitpunkt hatte ich noch keinen einzigen Patienten gesehen. Das würde auch noch etwas dauern, denn erstmal war der interdisziplinäre Rapport angesagt. Da wurde zwar über die Patienten gesprochen, aber ein Bild konnte ich mir immer noch nicht so richtig machen. Anschliessend durfte ich an der Oberarztvisite teilnehmen. 

Dies war nun die erste Annäherung an die stationären Patienten. Gut geschützt hinter dem Oberarzt, der Psychologin und der Pflege harrte ich nun den Dingen, die da kommen mögen. Als erstes sah ich einen Patienten, der an einer mittelgradigen Depression litt und zudem eine Persönlichkeitsakzentuierung im Bereich emotional instabile Persönlichkeitsstörung. Mir begegnete ein sympathisch wirkender Mann im mittleren Alter. Vielleicht etwas bedrückt. Aber niemals das, was ich erwartet hätte.  

In den nächsten Tagen nahm ich an vielen Hospitation teil. Darunter waren Einzelgespräche und Gespräche im Kernteam, welches aus einer Psychologin und einer Pflegeperson zusammensetzte. Mit jeder Person, die ich kennenlernen durfte, wich mein voreingenommenes Bild und machte Platz für ein menschlicheres, aufgeschlosseners Bild. 

Mein erster Kontakt, an den ich mich so richtig erinnern kann, war dann am dritten Tag. Meine Praktikumsbegleitung war nicht anwesend. Übergab mir jedoch eines ihrer Einzelgespräche. Ich war ziemlich nervös. Es handelte sich um eine Patientin, die bereits sehr klinikerfahren war und etwa zum 90igsten Eintritt zu uns kam. In ihrer Akte stand die Diagnose Emotional instabile Persönlichkeitsstörung vom Typ Borderline. Ich traf eine Frau etwas über 50 Jahre alt. Ihr Gesicht war mit Falten durchfurcht und sie hatte kurze, zum Bürstenschnitt aufgestellte Haare. Sie wirkte sehr freundlich. Da es für mich das erste Gespräch überhaupt in der Klinik war, wollte ich erstmal wissen, was sie dazu bewogen hat, stationär zu uns zu kommen. Die Frau gab breitwillig auskunft. Sie erzählte mir soviel aus ihrem Leben und genoss es sichtlich, dass ihr wieder einmal jemand mit voller Aufmerksamkeit zuhörte. Es entwickelte sich ein gutes gegenseitiges Gespräch, in dem sie auch mich über meine Funktion und meine Ausbildung ausfragte. Die Stunde, die ich mir dafür Zeit genommen habe, verging wie im Flug. Ich habe eine sichtlich zufrieden wirkende Frau aus dem Gespräch entlassen.

Weitere Gespräche folgten. Mit der Zeit entwickelte ich eine gewisse Übung und legte meine Unsicherheit nahezu komplett ab. Am Ende meines Praktikums gelang es mir immer mehr, egal an welchem Ort, Patienten in ein Gespräch zu verwickeln und ihre Stimmungslage zu erfragen. Teilweise machten wir gegenseitig Scherze auf dem Gang. Die Patienten kamen mit der Zeit immer mehr direkt auf mich zu, da sie in mir einen vollwertigen Gesprächspartner sahen. Sie vertrauten mir Dinge an, die sie nicht mal der direkten Bezugsperson anvertrauten (oder wenigstens sagten sie mir das). Am Ende meines Prkatikums fühlte ich mich auf der Station so wohl, dass es mir wie eine zweite Familie vorgekommen ist. Natürlich waren da auch Patienten, bei denen ich mich in acht nehmen musste. Bei einem paranoid Schizophrenen bewegten sich alle auf sehr dünnem Eis, da es sehr schnell gehen konnte, bis er dekompensierte und verbal aggressiv wurde. Ich bekam es jedoch immer so weit hin, dass dies in meinen Gesprächen nicht geschah.

Rückblickend schäme ich mich für meine Vorurteile. Kein einziges davon hat sich in der Praxis bewahrheitet. Ich lernte Menschen kennen, die ich sogar ein Bisschen in mein Herz geschlossen habe. Ich habe durch das Praktikum die Erfahrung machen können, dass ich den Kontakt zu Patienten sehr bereichernd für mein Leben finde. Dementsprechend viel es mir schwer, am Ende meiner Zeit wieder zu gehen. Nicht nur mein Abschied viel mir schwer, auch der Abschied von Patienten, die nach der Akutphase entweder nach Hause, oder in die Tagesklinik übergetreten sind vielen mir nicht immer leicht. Gelernt habe ich, dass in den meisten Fällen, eine empathische und zugewandte Haltung eine positive Wirkung auf das Gegenüber hat, auch wenn dieses psychisch noch so stark belastet ist. Diese Wissenslücke durfte ich durch den Sprung in die Praxis schliessen, denn so etwas lässt sich im Schlungsraum nicht erlernen. 

\paragraph{Gruppentherapie}
Neben diversen Therapieformen wie Musiktherapie, Bewegungstherapie, Ergotherapie, Einzelgespräche und andere Therapieformen, wurde einmal pro Woche eine Gruppentherapie durchgeführt. Diese wurde von der stationsunabhängigen Fachpsychologin der Klinik durchgeführt, welche von den Assistenzpsychologinnen abwechselnd in der der Co-Leitung unterstütz wurde. 

Während meinem Aufenthalt nahm ich jede Woche in der Co-Leitung an dieser Gruppentherapie teil. Die meiste Zeit als einziger Co-Leiter, da unsere Psychologinnen unter einer dauernden Unterbesetzung auf der Station litten und deshalb oft nicht an der Gruppe anwesend waren. 

Die Gruppe kam nicht bei allen Teilnehmer gut an. Sowohl die Co-Leiterinnen waren nicht alle über diese Form der Therapie erfreut und versuchten auch deshalb fern zu bleiben. Bei den Patienten stiess diese Form der Therapie oft auf Wiederstände. Einige äusserten, dass sie lieber Einzelgespräche führten, andere hatten einfach besseres zu tun, als sich die Sorgen und Nöte anderer Patienten anzuhören. Noch eine Gruppe war schlicht und einfach zu wenig gesund, um sich um 10 Uhr am Morgen in einen Raum ausserhalb der Station zu begeben.

Meine persönliche Meinung über die Gruppentherapie ist folgende: Ich bin ein grosser Fan von Gruppen. Da meine Einstellung die ist, dass die meisten pyschischen Probleme aus der Interaktion und Beziehungen mit anderen Menschen entstanden sind, ist die Gruppentherapie eine Methode, diese ungünstigen Interaktionen und Beziehungen korrektiv zu beeinflussen. In einer Gruppe können Erfahrungen in einem geschützten Rahmen gemacht werden, die überlagernd zu den negativen einen positiven Effekt haben können. Das Ziel dieser Gruppe war es, die Patienten dazu anzuregen, ihre Gefühle genau zu beschreiben und zu differenzieren. Dadurch können Themen entstehen, die alle mehr oder weniger gleichermassen betreffen und über die in der Gruppe diskutiert werden kann. Privat nahm ich während dem Studium während etwa eineinhlab Jahren wöchentlich an einer Gruppentherapie von 7 Personen teil. Ich bin also nicht mehr so ganz neu in diesem Thema. Mich faszinieren die Abläufe und die Möglichkeiten einer Gruppe. Ein grosser Fan bin ich natürlich von Irvin D. Yalom, einem amerikanischen Psychoanalytiker und Psychotherapeut, der mein Verständnis von Gruppentherapie nachhaltig prägte \cite{Yalom2010}.

In der Funktion als Co-Leitung durfte ich mich nach eigenem Gutdünken selber einbringen. Immer dazu gehörte ein persönliches Statement am Schluss, welches meine Sicht auf den Prozess widerspiegeln sollte. Natürlich war es auch meine Aufgabe, die Verlaufseinträge der einzelnen Patienten vorzunehmen. Dies stellte sich als die grösste Herausforderung dar, da unter Umständen bis zu 9 Stück anwesend waren und ich mir deren Prozess während der Stunde zu merken hatte. Natürlich ohne Notizen zu machen, da dies von der leitenden Psychologin so gewünscht wurde und mir als gute Übung diente. Dadurch konnte ich lernen, den Therapieprozess mit dem ständigen Mitschreiben nicht zu stören und meine gleichschwebende Aufmerksamkeit zu schulen. Am Anfang hatte ich grosse Mühe, mir die wichtigen Prozesse zu merken. Mit der Zeit konnte ich mir anhand der Vorstellung der einzelnen Gesichter der Patienten, die wichtigen Statements und Prozesse in Erinnerung rufen. 

Eines Morgens kam die Psychologin auf die Station und fragte mich, ob ich die Gruppe selber leiten möchte. Nach etwas Bedenkzeit willigte ich ein und machte meine erste richtige Erfahrung in diesem Bereich. Ich versuchte wie gewohnt auf die aktuellen Gefühle der Patienten Bezug zu nehmen. Ich hatte Glück, denn es waren nur 5 Patienten anwesend und diese waren sehr wohlgesonnen mit mir. Sie machten artig mit und betätigten sich aktiv an der anschliessenden Diskussion. Die leitende Psychologin griff dabei nur sehr wenig in den Prozess ein und half mir unterstützend dabei. Ich habe diese Erfahrung als sehr bereichernd empfunden. Sowohl die Leitung, als auch die Interaktion mit den Patienten. Ebenso die gemeinsame Leitung mit der Psychologin war von Respekt und Vertrauen mir gegenüber geprägt. Durch den Prozess, über die Gefühle in eine Diskussion einzusteigen, konnte ich meiner erste Gruppe leiten. Dennoch merkte ich, dass ich damit nicht wirklich zufrieden war. Irgendetwas fehlte. Ich hatte das Gefühl, dass die Gruppe noch nicht am Ende ihrer Möglichkeiten angelangt war. Nur wusste ich noch nicht genau was mir in der Gruppe fehlte. 

Kurz darauf beschloss das Team auf A2 neue Broschüren über die Therapieangebote zu erstellen. Diese sollen dazu dienen, den Patienten zusätzliche Informationen zum Angebot zu vermitteln. Darunter viel auch die Erstellung eines Flyers für die Gruppentherapie. Die leitende Psychologin bat mich, mit ihr zusammen ein paar Infos zusammenzutragen, die sie dann in den Flyer aufnehmen wollte. Zusammen diskutierten wir, was da alles rein soll. Ich gab an, dass ich es neben den Angaben über die Arbeit mit Gefühlen wichtig finden würde, die Patienten darüber zu informieren, dass die Gruppe ein gutes Instrument sei, korrigierende Erfahrungen in der Interaktion mit Mitmenschen zu machen. Zudem besprachen wir das Vorgehen von \citeA{Yalom2005} in seinem Buch "Im Hier und Jetzt". Ich bekundete mein Interesse an seinem Vorgehen und wollte ihre Meinung dazu hören. Die Psychologin war etwas zurückhaltend, vertrat sie nachweislich einen etwas anderen Ansatz. Für mich war dies natürlich in Ordnung. Wer war ich schon, der einen neuen Ansatz einzuführen gedenke. Doch ich habe mich getäuscht. In meiner letzten Gruppe, am letzten Tag meines Praktikums liess die Psychologin diesen Ansatz in die Gruppe einfliessen. Es lief hervorragend. Ob dies an der Grösse von leider nur drei Personen lag, oder dass die Patienten sehr gut mitmachten, die Stunde war ein Higlight in meinem Praktikum. Sogar die leitende Psychologin gab am Schluss bekannt, dass nicht alle Gruppen so toll funktionierten und diese eben sei eine Sternstunde in ihrer Geschichte von Gruppentherapie gewesen. 

Diese Erfahrung hat mich sehr gefreut. Konnte ich doch mit meiner Anregung wenigstens etwas im Rahmen der Gruppentherapie bewirken. Zudem bestärkte mich mein Gefühl, Gruppen in einem stationären Setting mit vielen Wechseln, anders zu halten als Gruppen, die über MOnaten bestand haben. Mein inneres Gefühl in richtung personnezentriertem Ansatz weiter zu gehen wurde gleichermassen gstärkt. Die Erfahrung in der Gruppentherapie gilt für mich als eine der reichhaltigsten im Rahmen meines Praktikums.

\paragraph{Psychoedukation}
Die Psychoedukation zählt zu den Angeboten an der Klinik, die auf allen Stationen in regelmässigen Abständen durchgeführt wird. Dabei werden die Themen entsprechend den einzelnen Schwerpunkten der jeweiligen Stationen leicht angepasst. Zum Beispiel wurde auf der Station A2 eine Vertiefung der affektiven Symptomen und einen Theorieteil Psychopharmaka im Bereich Antidepressiva vorgenommen. Die Psychoedukation umfasste sieben Einheiten, die über vier Wochen regelmässig durchgeführt wurden. Diese Kurse werden in der Regel von der zuständigen Aktiverungstherapeutin geleitet und folgen einem mehr oder weniger klinikweiten standardisierten Ablauf. Themen in diesen Kursen sind unter anderem das Psychosoziale-Modell, Frühwarnzeichen, unterschiedliche Symptome, ungünstiges Verhalten und Pharmakotherapie.

Meine Aufgabe bestand in der Co-Leitung dieses Kurses, indem ich die Materialien vorbereitete, für jede Kurseinheit ein Handout zusammenstellte, während dem Kurs die Rückmeldungen der Teilnehmer auf einem Flipchart notierte und diese Später in das nächste Handout einpfelgte. Zudem übernahm ich Teile der Theorie und förderte die aktive Teilnahme der Patienten, indem ich nachfragte und einzelne Patienten direkt zu ihrer Meinung aufforderte. 

In dieser Aufgabe konnte ich auf die Erfahrung des Psychologiestudiums zurückgreifen. Durch unzählige Gruppenarbeiten und Präsentationen sind wir es uns gewohnt vor Leuten zu referieren und interaktiv zusammenzuarbeiten. Die vorgestellten Modelle waren mir allesamt sehr gut bekannt und dementsprechend konnte ich mich als Co-Leiter aktiv einbringen. Mir gefiel diese Rolle sehr gut. Ich hatte das Gefühl, mich dabei entfalten zu können. Es freute mich, konnte ich dabei auf die unzähligen Lernerfahrungen des Studiums zurückgreiffen und quasi aus dem Vollen schöpfen. Die Rückmeldung der Leitung war dementsprechend sehr positiv. Ebenso die Rückmeldung der Patienten war positiv. Es schien, als ob sie von einem solchen Format der Therapie enorm profitieren konnten. 

Am Ende der Psychoedukation führten wir eine Feedbackrunde bei den Patienten durch. Wir wollten wissen, was ihnen am meisten gefallen hatte und was wir beim nächsten Mal verbessern könnten. Durchwegs positiv kam die Möglichkeit der aktiven Mitgestaltung bei den Patienten an. Sie schätzten es, kamen sie zu Wort und konnten sie dabei eine Expertenrolle bezüglich ihrer Krankheit übernehmen. Es scheint der Patienten wichtig zu sein, nicht nur passiv auf ein Urteil von einem Arzt oder Psychologen angewiesen zu sein, sondern sich viel mehr aktiv und mit mehr Kompetenz einbringen zu können.

Dies finde ich insbesondere ein wichtiger Punkt, da auch ich im Verlauf des Praktikums mich oftmals darüber gewundert habe, wie wenig den Patienten zugetraut wurde. Mein Verständnis scheint da mehr vom humanistischen Gedankengut geprägt zu sein, da ich tendentiell davon ausgehe, dass die Patienten mehr über ihre Krankheit wissen als jeder Aussenstehende es je könnte. Deshalb würde ich es sehr begrüssen, wenn in der Zukunft stärker auf die Bedürfnisse der Patienten eingegangen würde und ein Therapieangebot vermehrt Rücksicht auf die individuellen Bedürfnisse der Patienten eingehen würde. Ich werde auf diesen Punkt weiter unten näher eingehen (\prettyref{sec:Höhepunkte}).

\paragraph{Einzelgespräche}
Auf diesen Punkt möchte ich abschliessend von diesem Kapitel gerne etwas näher eingehen: Das Einzelgespräch. Wie bereits oben erwähnt, durfte ich bereits an meinem dritten Tag ein Einzelgespräch abhalten. Dabei fühlte ich mich zugegebenermassen etwas überfordert. Ich nahm jedoch die Herausforderung an und sprang ins kalte Wasser. Im Vorfeld wusste ich nicht viel über die mir zugeteilte Patientin. Nach der anfänglichen Panik, ausgelöst durch die Vorstellung alleine eine Patientin zu empfangen, löste sich meine Erstarrung und ich konnte etwas ruhiger an die Aufgabe herantreten. Ich hielt mir vor Augen, dass mir die Patientin von einer erfahrenen Psychologin zugeteilt worden ist. Also schloss ich daraus, dass sie schon wusste, was sie tat. Zweitens sah ich in dieser Aufgabe eine riesige Chance, als Praktikant etwas auszuprobieren, was ich später vielleicht nicht mehr so einfach konnte. Ich wollte so wenig wie möglich im Vofeld über die Patientin wissen. Sie sollte mir im Gespräch soviel sagen dürfen, wie sie es für richtig hielt. 

%----------------------------------------------
\subsubsection{Persönliche Höhepunkte und kritische Punkte} \label{sec:Höhepunkte}
\begin{itemize}
        \item Massenabfertigung vs. Individuelle Therapiegestaltung
        \item Klinische Behandlung mit Ziel auf Pat. Austritt
        \item Umgang mit Menschen durch die Erfahrung auf der Station (ich gehe viel offener auf Personen zu)
\end{itemize}
    
    
%----------------------------------------------
\subsubsection{Was es sonst noch aus meiner Sicht zu sagen gib}
\begin{itemize}
    \item Ein Psychologe unter Psychiatern. Die psychologische Arbeit in einem medizinalen Umfeld.
    \item Heilung oder Schadensbegrenzung?
    \item Ev. Team
    \item Ev. Hierarchie, Stellung, Akzeptanz als Praktikant der ZHAW
    \item Kennengelernte Störungsbilder (Borderline, Depression, Bipolare-Störung, Schizoaffektive-Störung, Manische-Episode, Suchterkrankung, Hebephrene-Schizophrenie)
\end{itemize}

%##############################################
\subsection{Intervisionsreflexion} \label{sec:Intervision}
\begin{itemize}
    \item Rahmen (Skype, Personen, Vorgehen)
    \item Eigener Fall - Reflecting Team (Pat. Vorstellen, Mühe mit, Rückmeldungen Team)
\end{itemize}

\subsection{Fazit}
TBD

% Literaturverzeichnis
\begin{flushleft}
\nocite{*}
\bibliography{Literaturverzeichnis}{}
\end{flushleft}

\end{document}
