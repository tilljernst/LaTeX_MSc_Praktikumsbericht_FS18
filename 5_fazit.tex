Welche Erkenntnisse nehmen ich nun am Ende des Kurses ER4-1 Praktikum 2 mit? Eine nicht ganz einfach zu beantwortende Frage, da die drei Monate inklusive Gruppenintervision viele Situationen bereithielt, aus denen ich lernen und Erfahrungen sammeln konnte.

Auf alle Fälle möchte ich im Hinblick auf eine mögliche Karriere im klinischen Bereich auf die kennengelernten Menschen und deren Störungsbilder eingehen. In den drei Monaten kam ich in den Kontakt mit unterschiedlichen Diagnosen, von Borderline über Bipolare-Störung zu paranoider Schizophrenie und anderen. Dies hat mein Verständnis von psychiatrischen Diagnosen und Störungsbilder massgeblich geprägt. Können wir Diagnosen stellen, ohne den Menschen dahinter nicht aus den Augen zu verlieren? Ist eine Diagnosen nicht vielmehr eine Antwort der Psychiatrie auf eine Hilflosigkeit im Umgang mit Patienten? Es scheint einfacher eine diagnostizierte Störung  zu behandeln, als den Mensch dahinter zu verstehen.

Im Zusammenhang mit den Patienten durfte ich die überaus wertvolle Erfahrung des menschlichen Kontakts, aus Sicht eines Psychologen machen. In den Augen der Patienten war ich ein Psychologe und als solcher wurde ich behandelt. Die Erfahrung, die ich durch die vielen direkten Gespräche machen durfte, bestärkten mich auf meinem Weg, therapeutisch tätig zu werden. Ob dies im psychiatrischen Setting stattfinden wird, das ist noch offen.

Die Klinik mit ihren - ich bin verleitet \glqq rigide\grqq{} vorne dran zu stellen - Strukturen, hinterliessen einen zwiespältigen Eindruck. Die Mediziner als Oberhäupter der Klinik. Auf deren Wohlwollen angewiesen zu sein. Das beschäftigt mich seit meinem Praktikum zunehmends. Meine Erfahrung in diesen drei Monaten war es, dass die Psychologie als ein Fachgebiet der Medizin angeschaut wird. Die Psychologinnen und Psychologen als Psychiater und Psychiaterinnen geringerer Ordnung und immer einem Arzt unterstellt. Keine eigenständige Disziplin, die sie sein sollte. Gleichberechtigt und nicht besser. Aber zumindest auf gleicher Augenhöhe operierend. Diesen Misstand gilt es zukünftig zu vermeiden, wenigstens zu verringern. Dazu werden fähige Psychologinnen und Psychologen benötigt, die unsere Profession auf eine weitere Stufe heben. Dies muss aus meiner Sicht mit dem Verständnis der Psychologie als eigenständiges Fach- und Kompetenzgebiet erfolgen. Es kann nicht sein nur weil Ärztemangel herrscht, psychologisch geschultes Personal in die Bresche springt. Ich möchte meine Profession als solche akzeptiert sehen.

Am Ende meines Praktikums und am Ende meines Psychologiestudiums erkenne ich eine grosse Freude über die Bereicherung der letzten Jahre, aber auch eine grosse Frustration gegenüber dem immensen Aufwand (finanziell und emotional). Um nicht mit dem negativ angehauchten Gedanken zu Enden erst die pessimistische Sicht und anschliessend die Positive: Riesige Kosten im Sinne von Aufwand, Zeit und Geld, welche ich in den letzten acht Jahre aufbringen musste. Doch bleibt eine gewisse Ernüchterung bezüglich der beruflichen Möglichkeiten, ohne therapeutische Ausbildung. Die Stellen sind vorwiegend im postgradualen Bereich angesiedelt und sind in der Regel schlecht bezahlt. Der weitere kostspielige Weg zum Therapeuten, der anschliessend niemals die Kompetenz eines Mediziners erreichen wird (bezogen auf die aktuelle Lage). Eine riesige Investition, die niemals zu Ende ist und für die die Anerkennung im institutionellen Bereich vorerst zu fehlen scheint. Ich frage mich ernsthaft, hat sich die Mühe gelohnt? Oder liess ich mich von gut formulierten Verkaufsargumenten bezüglich Studium blenden? Bin ich einfach 15 Jahre zu alt und stehen mir deshalb mein Alter und meine Familie im Weg? 

Doch, es hat sich gelohnt. Nicht nur habe ich meine Frau im Studium kennengelernt, mit der ich nun einen gemeinsamen Sohn habe. Heute sehe ich die Investition in meine persönliche Art zu denken. Die Schulung des Geistes und des Verstandes. Ein Thema gefunden zu haben, in welchem es Handlungsbedarf bezüglich Akzeptanz gibt, aber auch weiterhin viele unerforschte Bereiche, die es zu entdecken gibt. Natürlich bezahlt mir dies direkt keinen Lohn. Doch sollte es mit meinem Verstand möglich sein, eine Nische für mich zu finden, die mir entspricht und in der ich meine Fähigkeiten einbringen werde. 

Das Praktikum hat mir am Schluss meines Studiums die Möglichkeit gegeben, den zurückgelegten Weg zu reflektieren und meinen aktuellen Standpunkt wahrzunehmen. Die gemachten Erfahrungen sind ungemein wichtig für mich und meinen weiteren Lebensweg, auch wenn der Aufwand ein hoher war. Ja, ich würde es noch einmal so machen. Auch wenn ich mir wünschte, dass es früher passiert wäre. Dies gehört nun zu meiner Biographie, worüber ich froh und stolz bin.

