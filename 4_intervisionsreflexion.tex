\subsubsection{Einleitung}
In diesem Teil werde ich auf den Prozess der Intervisionsgruppe  eingehen. Darin werden in einem ersten Teil die Rahmenbedingungen erläutert, anschliessend werde ich unser Vorgehen anhand eines konkreten Falls beschreiben und mit einer Reflexion des gesamten Intervisionsprozesses abrunden. 

\subsubsection{Rahmenbedingungen}
Gemäss den Praktikumsrichtlinien des konsekutiven Masterstudiengangs Angewandte Psychologie gehört zum Leistungsnachweis eine Intervisionsreflexion. Diese basiert auf einer selbständig organisierten Intervisionsgruppe, die vorzugsweise während dem Modul ER4-1 Praktikum 2 abgehalten wird und aus 3 bis 4 Mitstudierenden besteht.

Da die Studierenden für die Bildung der Gruppen selber zuständig sind, habe ich Ende 2017 via Klassenmail einen Aufruf für die Bildung einer Intervisionsgruppe gestartet. Darin beschrieb ich den gewünschten zeitlichen Rahmen und die Vertiefungsrichtung, da ich vorzugsweise Mitstudierende aus diesem Feld ansprechen wollte. Auf die Mail haben sich sieben Mitstudierende gemeldet. Da die Gruppengrösse auf maximal vier Personen beschränkt wurde, konnten wir zwei Gruppe bilden. Die gesamte Koordination fand 
mittels WhatsApp-Gruppennachrichten statt.

Als die Gruppen gebildet waren, traf sich unsere Gruppe nach einer regulären Vorlesung im HS17 zu einer ersten Vorbesprechung im Toni-Gebäude. Für unsere regelmässigen Treffen wollten wir ein neues Medium nutzen, da wir aus unterschiedlichen Gebieten der Schweiz kommen und uns aus zeitlichen Gründen nicht vor Ort treffen konnten (viele von uns leisteten ein 100\% Praktikum). Wir beschlossen unsere Intervision via Skype-Gruppenvideochat durchzuführen. Dieses Medium schien uns passend und gut durchführbar. 

Die Termine für unsere Treffen wurden im Voraus festgelegt. Wir trafen uns etwa alle zwei Wochen über Skype. Die Themen wurden meist im Vorfeld über mobile Textnachrichten eingebracht und von einer sich abwechselnden Konferenz-Leiterin in die Diskussion eingebracht. Wir orientierten uns stark am Vorgehen des \textit{Reflecting Teams} \cite{ReflectingTeam2018}. Dabei schauten wir, dass alle mindestens einmal die Rolle der Moderation und der Fallvorstellung ausübten. Falls eine Person eine weitere Methode der Intervision ausprobieren wollte, sprach sie dies mit der Gruppe im Vorfeld ab. Dabei wurde die Kommunikation und der Austausch von Dokumenten über die eigens dazu erstellte WhatsApp-Gruppe abgewickelt.

\subsubsection{Reflexion eines eigenen Falls}
Die Intervisionsgruppe nutzte ich, um einen für mich  schwierigen Fall mittels Reflecting Team zu besprechen. Diese Intervisionssitzung fand am Anfang meines Praktikums statt. Den Fall habe ich ausgewählt, weil dieser mich emotional stark beschäftigte. Dabei ging es mir um die Rahmenbedingungen, wie dieser Patient behandelt wurde und wie er zu uns auf die Station gekommen ist. 

Den Ablauf legten wir in etwa so fest: Zu Beginn schilderte ich in etwa 10 Minuten den Fall. Anschliessend bestand die Möglichkeit auf Seiten der Intervisionsmitgliedern Verständnisfragen zu stellen. Danach formulierte ich Fragen und Anliegen an die Gruppe, welche darauf Hypothesen bildete. Während dieser Zeit nahm ich mich aus der Gruppe raus und hörte zu. Nach der Diskussion und Hypothesenbildung beteiligte ich mich wieder aktiv an der Intervision, indem ich ein Feedback, in Form einer Resonanz, der Gruppe zurückgab. Am Schluss diskutierten wir über den Gruppenprozess hinsichtlich der Gruppendynamik und dem Output.

\paragraph{Fall S im Schnelldurchlauf}
Der Patient S wurde zu uns mittels FU (Fürsorgerische Unterbringung) zugewiesen. Anscheinend sei er bei einer Verhandlung mit der KESB (Kinder und Erwachsenenschutzbehörde) verbal aggressiv und drohend geworden. Es handelte sich dabei um einen bekannten Patienten, bei dem bei den letzten Aufenthalten die Diagnose paranoide Schizophrenie gestellt wurde. Der Patient wurde am Abend mit zwei Polizistinnen und zwei Rettungssanitäterinnen auf unsere Station gebracht. Er wurde umgehen ins Isolationszimmer verlegt. Der Mann liess sich gut ins Zimmer begleiten und folgte den Anweisungen des Pflegepersonals. Er schien vordergründig nicht aggressiv zu sein, jedoch lag für mich eine merklich spürbare Spannung in der Luft. Das Pflegepersonal war sehr vorsichtig und entspannte sich erst, als die Iso-Türe verschlossen wurde.

In den folgenden Tagen verbesserte sich der Zustand von Herrn S merklich. Die Isolation wurde gelockert und der Patient durfte sich auf der Station frei bewegen. Kurze Zeit später musste die Isolation wieder verschärft werden, da der Patient vermehrt aggressiv gegenüber den Mitpatienten und dem Pflegepersonal aufgetreten ist. 

Die KESB Verhandlung wurde in der Klinik nachgeholt, wobei sich der Patient den Verhältnissen entsprechend adäquat verhalten konnte. 

Meine Schnittpunkte mit dem Patienten waren zu Beginn seines Aufenthaltes die Betreuung im Isolationszimmer. Das heisst, ich führte den Atemalkoholtest durch, wechselte das Handurinal, gab ihm Wasser und überwachte ihn beim Rauchen. Etwas später, als die Isolation gelockert wurde, ging ich mit dem Patienten auf dem Klinikareal spazieren. 

\paragraph{Mein Anliegen}
Da mir dieser Fall emotional nahe ging, wollte ich mittels Reflecting Team Hinweise für einen gesunden Umgang mit schwierigen Fällen erarbeiten. Als Psychiatrieneuling mangelte es mir an Erfahrung  und Bewältigungsstrategien. Mein Ziel war es, andere Blickwinkel zu erlangen. Des Weiteren beschäftigte mich der Umgang mit dem Isolierzimmer im allgemeinen. Diese Punkte wollte ich in der Intervision besprochen haben. 

\paragraph{Rückmeldung Team}
Das Team meldete zurück, dass im Bezug zu meiner mangelnden Erfahrung im Psychiatriekontext und einem solch überfordernden Fall ein Austausch mit dem Stationsteam und den Vorgesetzten sehr zu empfehlen sei. Dieser Fall sei aus der Sicht der Gruppe schwerwiegend und dazu könnte eine Nachbesprechung überaus Sinn ergeben.

Zudem diskutierte die Gruppe über die Verlegung eines Patienten in ein Iso-Zimmer, was ein fundamentaler Eingriff in die Menschenwürde darstelle. Damit werde über einen Menschen gegen seinen Willen verfügt. Die Gruppe stellte sich die Frage, wo die Grenzen gezogen werden und wer am Ende die Entscheidungsgewalt für einen solchen Eingriff hat.

Weiter diskutierte die Gruppe über meinen Umgang mit dem Patienten und ob es wohl sinnvoll wäre, dies direkt mit dem Patienten anzuschauen und ihn direkt zu befragen. 

Zudem solle ich doch seine Fallgeschichte vertieft studieren, um mir ein besseres Bild von dem Patienten machen zu können.

Grundsätzlich spiegelte mir die Gruppe, dass ich mich als Person zeigen solle und nicht versuchen soll eine Fassade zu wahren. Ein proaktives Zugehen auf die Beteiligten könnte für mich hilfreich sein, indem ich sie zu einer solch schwierigen Situation über ihre Meinung befragen würde. 

\paragraph{Was konnte ich mitnehmen}
Die meisten Rückmeldungen erachtete ich als hilfreich und setzte sie in den weiteren Tagen um. Bis auf den Vorschlag, den Patienten direkt zu fragen. Das erachtete ich in diesem Fall als weniger hilfreich. Der Patient war zu instabil und war nur mittels starker Begrenzung absprachefähig.  

\subsubsection{Reflexion über den Intervisionsprozess}
Durch die Intervision war es mir möglich, meine aktuelle Situation auf der Station mittels Fallvorstellung besser greifbar zu machen. Dabei wurde mir bewusst, dass ich meiner Überforderung keinen Platz gegeben hatte. Ich war so tief im Arbeitsprozess vertieft, dass ich verdrängte, wie es mir dabei emotional ergangen ist. Durch die Hilfe der Intervision gelang es mir, mich einen Schritt aus der aktuellen Situation zu entfernen und eine Metaebene einzunehmen. Durch die durchwegs positiven Rückmeldungen bekam ich Ideen in die Hand, wie mit der schwierigen Situation umzugehen. 

Weiter zeigte mir die Arbeit in der Intervision weitere Perspektiven auf. Das hat mir geholfen, nicht nur die Perspektive zu ändern, sondern vielmehr sie zu erweitern, um ein neues Ganzes daraus zu bilden. Ich denke, dass mir die Gruppe zu einem besseren Verständnis verholfen hat, oder wenigsten einen Schubs in die Richtung gegeben hat.

Positiv zu erwähnen ist die Zusammensetzung der Intervisionsgruppe aus externen Studierenden. Den Mehrwert sehe ich in den Themen, die mit der Intervisionsgruppe besprochen werden konnten. Bei einem internen Coaching, auch wenn die zuständige Person noch so professionell ist, können nicht alle Themen offen besprochen werden. Aus rein strategischen Überlegungen würde ich Klinik kritische Themen nicht intern behandeln. Durch die Besetzung der Intervisionsgruppe durch externe, unabhängige Personen, können durchaus psychiatriekritische Themen behandelt werden.

Des Weiteren möchte ich das Gefühl, das ich nach der Intervision mitgenommen habe, nicht unerwähnt lassen. Fühlte ich mich vor unserem Treffen reichlich verstört über die Erlebnisse auf der Station, so konnte ich nach der Sitzung mit einem erleichterten Gefühl die Klinik verlassen. Die Gruppe hat mich beschwingt und geholfen Unstimmigkeiten und negative Gefühle anzunehmen, ernst zu nehmen und zu behandeln. Den Fokus auf Aspekte legen dürfen, die im normalen Arbeitsalltag oftmals keinen Platz finden oder für die schlicht und ergreifend während einem hektischen Klinikalltag keine Zeit ist. 

Eine letzte Reflexion soll unserem Vorgehen gewidmet werden: Das Treffen über Skype. Wir wussten im Vorfeld nicht, wie und ob eine Intervision über Skype funktionieren würde. Wir hatten alle das Modul der Distanzberatung absolviert und konnten uns in etwa vorstellen, was uns erwarten würde. Trotzdem konnten wir uns nicht sicher sein. Denn diesmal war es kein Zweiergespräch. Es handelte sich um eine Gruppe von vier Personen, die sich alle gleichzeitig über Internet hören und sehen würden. Die Skepsis legte sich sehr rasch während unserem ersten Treffen. Nachdem die anfänglichen technischen Schwierigkeiten behoben waren - wir alle mussten zu einem Headset wechseln, da die Tonqualität mit dem Freisprechmikrofonen zu schlecht war und das Gespräch von störenden Rückkopplungen geprägt war - konnten wir die Vorzüge der Onlinekommunikation ausschöpfen. Wie wir schon in der Distanberatung erfahren konnten, wurde in unserem Onlinegespräch viel schneller auf den Kern der Sache eingegangen. Wir benötigten keine grosse Zeit im Vorfeld für das Einstimmen. Nach knappen 5 Minuten des Austauschens stiegen wir direkt in den eigentlichen Prozess ein. Zudem war für mich die Ablenkbarkeit durch das Medium Skype stark reduziert. Ich konnte mich bei der Fallvorstellung und der Schilderung meiner Anliegen besser konzentrieren. Ich hatte das Gefühl besser auf den Punkt zu kommen. Aus dieser Gründen würde ich wieder zu dieser Methode greifen. Insbesondere dann, wenn sich das finden eines gemeinsamen Treffpunkts als Hinderniss darstellt. 

\subsubsection{Schlussbemerkung zur Intervision}
Auch wenn ich mich zu Beginn etwas über den Zusatzauftrag genervt habe, konnte ich von der Intervision profitieren. Ob die Intervision zukünftig in diesem zeitlichen Rahmen ausfallen soll, das sei mal dahingestellt. Es ist eine Herausforderung die Zeit zu finden, in der alle Teammitglieder sich treffen können. Ob sich der Aufwand lohnt? Für mich hat er sich gelohnt. Ich konnte gut von den Gesprächen profitieren. Auch wenn ich primär auf meinen Fall eingegangen bin, so konnte ich auch von den Beiträgen der übrigen Teilnehmern profitieren. Mir reicht es schon, wenn ich meine Gedanken einem interessierten Zuhörer verbalisieren und verständlich rüberbringen kann. Dies hat für mich bereits therapeutischen Charakter.

Was ich auf alle Fälle mitnehme oder mitnehmen möchte, ist die Achtsamkeit zukünftig über belastenden Themen zu sprechen. Mit den Mitarbeitern oder allenfalls mit den Vorgesetzten. Es hilf mir mehr, wenn ich Farbe bekenne und anschliessend besser mit meinen Fragen umgehen kann, als wenn ich die Warnzeichen ignoriere. Diese Intervisionsarbeit hat mir die Notwendigkeit für den aktiven Diskurs aufgezeigt. Vor allem in einem sozialen Bereich wo wir uns tagtäglich mit den Nöten anderen Menschen auseinandersetzen. Da könnte wohl die Achtsamkeit auf die eigenen Nöten in der Hitze des Gefechts schneller unter gehen als einem lieb ist.
