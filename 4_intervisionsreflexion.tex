\subsubsection{Rahmenbedingungen}
Gemäss den Praktikumsrichtlinien des konsekutiven Masterstudiengangs Angewandte Psychologie gehört zum Leistungsnachweis eine Intervisionsreflexion. Diese wird basierend auf einer selbständig organisierten Intervisionsgruppe erstellt. 

Ende 2017 habe ich via Klassenmail einen Aufruf für die Bildung einer Intervisionsgruppe gestartet. Ich beschrieb in diesem Aufruf mein Praktikum und für welchen Zeitraum ich gerne eine Gruppe bilden würde. Auf diese Mail haben sich etwa sieben Mitstudierende gemeldet. Da die Gruppengrösse auf maximal vier Personen beschränkt wurde, konnten wir zwei Gruppe zu drei und vier Personen bilden. Die gesamte Koordination nahmen wir mittels Whatsapp-Gruppennachrichten vor.

Als die Gruppen gebildet waren, traf sich unsere Gruppe nach einer regulären Vorlesung im HS17 zu einer ersten Vorbesprechung. Darin wurde beschlossen, dass wir unsere Intervision via Skype-Gruppenvideochat durchführen würden. Wir wollten ein neues Medium nutzen und uns über Video treffen. Da wir alle an unterschiedlichen Orten in der Schweiz wohnen, kam uns diese Variante sehr entgegen. Ein physisches Treffen war insofern schwierig, da einige von uns ein 100\% Praktikum absolvierten.

Die Termine für unsere Treffen wurden im Voraus festgelegt. Wir haben uns etwa alle zwei Wochen über Skype verbunden. Die Themen wurden von den einzelnen Mitglieder eingebracht. Wir orientierten uns stark am Vorgehen des Reflecting Teams. Dabei schauten wir, dass alle mindestens einmal die Rolle der Moderation und der Fallvorstellung ausübten. Falls eine Person eine weitere Methode der Intervision ausüben wollte, sprach sie dies mit der Gruppe im Vorfeld ab. Dabei wurde die Kommunikation und der Austausch von Dokumenten über die eigene Intervisions-Whatsapp-Gruppe abgewickelt.

\subsubsection{Reflexion eines eigenen Falls}
Die Intervisionsgruppe nutzte ich, um einen für mich  schwierigen Fall mittels Reflecting Team zu besprechen. Diese Intervisionssitzung fand am Anfang meines Praktikums statt. Den Fall habe ich ausgewählt, weil dieser mich emotional stark beschäftigte. Dabei ging es mir nicht primär um das Störungsbild des Patienten, sondern um die Rahmenbedingungen, wie dieser Patient behandelt wurde und wie er zu uns auf die Station gekommen ist. 

Vom Ablauf einigten wir uns so, dass ich in etwa 10 Minuten den Fall schilderte. Anschliessend bestand die Möglichkeit auf Seiten der Intervisionsmitgliedern Verständnisfragen zu stellen. Danach formulierte ich Fragen und Anliegen an die Gruppe, welche ich behandelt haben wollte. Diese bildete Hypothesen, basierend auf der Fallgeschichte und meinen Wünschen. Während dieser Zeit nahm ich mich aus der Gruppe raus und hörte primär zu. Nach der Diskussion und Hypothesenbildung der Gruppe nahm ich wieder aktiv an der Intervision teil. Es folgte die Resonanz von meiner Seite. Was hatte ich alles aufgenommen und welche Punkte wollte ich weiter verfolgen. Am Schluss diskutierten wir über den Gruppenprozess hinsichtlich der Gruppendynamik und dem Output.

\paragraph{Fall S im Schnelldurchlauf}
Der Patient S wurde zu uns mittels FU (Fürsorgerische Unterbringung) zugewiesen. Anscheinend ist er bei einer Verhandlung mit der KESB (Kinder und Erwachsenenschutzbehörde) verbal aggressiv und drohend geworden. Es handelt sich dabei um einen bekannten Patienten, bei dem bei den letzten Aufenthalten die Diagnose paranoide Schizophrenie gestellt wurde. Der Patient wurde am Abend mit zwei Polizistinnen und zwei Rettungssanitäterinnen auf unsere Station gebracht, worauf er umgehen ins Isolationszimmer verlegt wurde. Der Mann liess sich gut ins Zimmer begleiten und folgte den Anweisungen des Pflegepersonals. Er schien vordergründig nicht aggressiv zu sein, jedoch lag eine merklich spürbare Spannung in der Luft. Das Pflegepersonal war sehr vorsichtig und konnte erst durchatmen, als die Türe verschlossen wurde.

In den folgenden Tagen verbesserte sich der Zustand von Herrn S. Die Isolation wurde gelockert und der Patient durfte sich auf der Station frei bewegen. Kurze Zeit später musste die Isolation wieder verschärft werden, da der Patient vermehrt aggressiv gegenüber den Mitpatienten und dem Pflegepersonal aufgetreten ist. Die KESB Verhandlung wurde in der Klinik nachgeholt, wobei sich der Patient den Verhältnissen entsprechend adäquat verhalten konnte. 
Meine Schnittpunkte mit dem Patienten waren zu Beginn die Betreuung im Isolationszimmer. Das heisst, ich führte den Atemalkoholtest durch, wechselte das Handurinal, gab ihm Wasser und überwachte ihn beim Rauchen. Etwas später, als die Isolation gelockert wurde, ging ich mit dem Patienten auf dem Klinikareal spazieren. 

\paragraph{Mein Anliegen}
Da mir dieser Fall nahe ging, wollte ich mittels Reflecting Team Hinweise für einen gesunden Umgang mit schwierigen Fällen erarbeiten. Als Psychiatrieneuling mangelte es mir an Erfahrung  und Bewältigungsstrategien. Mein Ziel war es, andere Blickwinkel zu schaffen. Des Weiteren beschäftigte mich der Umgang mit dem Isolierzimmer, welches ich in der Intervision besprochen haben wollte. 

\paragraph{Rückmeldung Team}
Das Team meldete zurück, dass im Bezug zu meiner mangelnden Erfahrung im Psychiatriekontext und einem solch überfordernden Fall ein Austausch mit dem Stationsteam und den Vorgesetzten sehr zu empfehlen wäre. Zudem diskutierte die Gruppe über die Verlegung eines Patienten in ein Iso-Zimmer, was ein fundamentaler Eingriff in die Menschenwürde sei und damit über einen Menschen verfügt werde. Die Gruppe fragte sich, wo die Grenzen gezogen werden und wer am Ende die Entscheidungsgewalt für einen solchen Eingriff hat. Auch betreffend dieser Fragen solle ich mich zeigen und keine Fassade wahren, sondern proaktive auf die Beteiligten zugehen und sie dazu befragen. 

Weiter diskutierte die Gruppe über meinen Umgang mit dem Patienten und ob es wohl sinnvoll wäre, dies direkt mit dem Patienten anzuschauen und ihn dazu direkt zu befragen. Zudem solle ich doch seine Fallgeschichte vertieft studieren, um mir ein besseres Bild von dem Patienten machen zu können.

\paragraph{Was konnte ich mitnehmen}
Die meisten Rückmeldungen erachtete ich als hilfreich und setzte sie in den weiteren Tagen um. Bis auf den Vorschlag, den Patienten direkt zu fragen. Das erachtete ich in diesem Fall als weniger hilfreich. Der Patient war zu instabil und war nur mittels starker Begrenzung absprachefähig. 

Durch meine Öffnung gegenüber dem Team machte ich die Erfahrung, dass es vielen erfahrenen Fachpersonen ähnlich ergeht. Sie sprechen einfach nicht mehr so viel darüber, was aus meiner Sicht sehr schade ist. Es wird dadurch ein Tabu geschaffen, welches nicht sein müsste. Dadurch, dass ich meine Unsicherheit mit dem Theam teilte, konnten weitere Personen über ihre Bedenken und Unsicherheiten sprechen. Dies trug vielleicht auch in gewisser Weise dazu bei, dass in der nächsten Fallsupervision des Stationsteams, dieser Fall besprochen wurde und ähnliche Themen wie Isolation und Zwangsmedikation behandelt wurden.

\subsubsection{Reflexion über den Intervisionsprozess}
Durch die Intervision war es mir möglich, meine aktuelle Situation auf der Station mittels Fallvorstellung für mich besser greifbar zu machen. Dabei wurde mir bewusst, dass ich meiner Überforderung, durch die vielen unterschiedlichen Tätigkeiten an einem für mich neuen Arbeitsort, keinen Platz gegeben hatte. Ich war so tief im Arbeitsprozess drinnen, in dem ich nicht mehr objektiv sehen konnte, wie es mir dabei emotional ergangen ist. Durch die Hilfe der Intervision gelang es mich einen Schritt aus der aktuellen Situation zu entfernen und eine Metaebene einzunehmen. Durch die durchwegs positiven Rückmeldungen bekam ich Ideen in die Hand, wie mit der schwierigen Situation umzugehen. 

Weiter zeigte mir die Arbeit in der Intervision weitere Perspektiven auf. Dadurch gelang es mir weitere Sichtweisen über die einzelnen Gruppenteilnehmer zu erlangen und diese mir genauer anzuschauen. Das hat mir geholfen, nicht nur die Perspektive zu ändern, sondern vielmehr sie zu erweitern, um ein neues Ganzes daraus zu bilden. Ich denke, dass mir die Gruppe zu einem besseren Verständnis verholfen hat, oder wenigsten einen Schubs in die Richtung gegeben hat, mich aktiv um weitere Informationen zu kümmern.

Positiv zu erwähnen ist die Zusammensetzung der Intervisionsgruppe. Ich sehe es positiv, dass die Mitglieder alle aus dem Studium waren und nicht direkt mit der Stelle in Verbindung standen. Ich hatte, wie im Praktikumsbericht erwähnt, zusätzlich ein hausinternes Coaching bei einer erfahrenen Psychologin. Den Mehrwert sehe ich in den Themen, die mit der Intervisionsgruppe besprochen werden konnten. Bei einem internen Coaching, auch wenn die zuständige Person noch so professionell ist, können nicht alle Themen offen besprochen werden. Schon ein leichtes Interesse an einer Anstellung in diesem Bereich würde mich abhalten, gewissen Themen ausführlich besprechen zu wollen. Vielleicht müssten ja genau diese Themen besprochen werden. Doch aus rein strategischen Überlegungen, würde ich Klinik kritische Themen nicht unbedingt intern behandeln. Durch die Besetzung der Intervisionsgruppe durch externe, unabhängige Personen, können durchaus psychiatriekritische Themen behandelt werden.

Des Weiteren möchte ich das Gefühl, das ich nach der Intervision mitgenommen habe, nicht unerwähnt lassen. Fühlte ich mich vor unserem Treffen reichlich verstört über die erlebten Vorkommnisse auf der Station, so konnte ich nach der Sitzung mit erhobenen Hauptes die Klinik verlassen. Die Gruppe hat mich beschwingt und geholfen Unstimmigkeiten und negative Gefühle anzunehmen, ernst zu nehmen und zu behandeln. Den Fokus auf Aspekte legen dürfen, die im normalen Arbeitsalltag oftmals keinen Platz finden oder für die schlicht und ergreifend während einem hektischen Klinikalltag keine Zeit ist. 

Eine letzte Reflexion soll unserem Vorgehen gewidmet werden: Das Treffen über Skype. Wir wussten im Vorfeld nicht, wie und ob eine Intervision über Skype funktionieren würde. Würden wir die physische Anwesenheit der restlichen Gruppenteilnehmer nicht vermissen? Wir hatten alle das Modul der Distanzberatung absolviert und konnten uns in etwa vorstellen, was uns erwarten würde. Trotzdem konnten wir uns nicht sicher sein. Denn diesmal war es kein Zweiergespräch oder ein Zweierchat. Es handelte sich um eine Gruppe von vier Personen, die sich alle gleichzeitig über Internet hören und sehen würden. Die Skepsis legte sich sehr rasch während unserem ersten Treffen. Nachdem die anfänglichen technischen Schwierigkeiten behoben waren - wir alle mussten zu einem Headset wechseln, da die Tonqualität mit dem Freisprechmikrofonen zu schlecht war und das Gespräch von störenden Rückkopplungen geprägt war - konnten wir die Vorzüge der Onlinekommunikation ausschöpfen. Wie wir schon in der Distanberatung erfahren konnten, wurde in einem Onlinegespräch viel schneller auf den Kern der Sache eingegangen. Wir benötigten keine grosse Zeit im Vorfeld für das Einstimmen. Nach knappen 5 Minuten des Austausch stiegen wir direkt in den eigentlichen Prozess ein. Zudem war für mich die Ablenkbarkeit durch das Medium Skype stark reduziert. Ich konnte mich bei der Fallvorstellung und der Schilderung meiner Anliegen besser konzentrieren. Ich hatte das Gefühl besser auf den Punkt zu kommen. Aus dieser Gründen würde ich wieder zu dieser Methodenwahl greifen. Insbesondere dann, wenn sich das finden eines gemeinsamen Treffpunkts als Hinderniss darstellt. 

Auch wenn ich mich zu Beginn etwas über den Zusatzauftrag genervt habe, konnte ich von der Intervision profitieren. Ob die Intervision zukünftig in diesem zeitlichen Rahmen ausfallen soll, das sei mal dahingestellt. Es ist eine Herausforderung die Zeit zu finden, in der alle Teammitglieder sich treffen können. Ob sich der Aufwand lohnt? Für mich hat er sich gelohnt. Ich konnte gut von den Gesprächen profitieren. Auch wenn ich primär auf meinen Fall eingegangen bin, so konnte ich auch von den Beiträgen der übrigen Teilnehmern profitieren. Mir reicht es schon, wenn ich meine Gedanken einem interessierten Zuhörer verbalisieren und verständlich rüberbringen muss. Dies hat für mich bereits therapeutischen Charakter.

Was ich auf alle Fälle mitnehme oder mitnehmen möchte ist die Achtsamkeit, in meinem zukünftigen Arbeitsbereich über belastenden Themen zu sprechen. Mit den Mitarbeitern oder allenfalls mit den Vorgesetzten. Es hilf mir mehr, wenn ich Farbe bekenne und anschliessend besser mit meinen Fragen umgehen kann, als wenn ich die Warnzeichen ignoriere und mit deren Umgang Mühe habe. Diese Intervisionsarbeit hat mir die Notwendigkeit für den aktiven Diskurs aufgezeigt. Vor allem in einem sozialen Bereich wo wir uns tagtäglich mit den Nöten anderen Menschen auseinandersetzen. Da geht wohl die Achsamkeit auf die eigenen Nöten in der Hitze des Gefechts schneller unter als einem lieb ist.
