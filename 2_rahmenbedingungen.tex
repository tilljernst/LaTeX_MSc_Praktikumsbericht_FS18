\subsection{Rahmenbedingungen und Fakten} \label{sec:Rahmenbedingungen}

%----------------------------------------------
\subsubsection{Ort, Dauer und Entlöhnung} 
Das Praktikum absolvierte ich an der \textit{Klinik für Psychiatrie und Psychotherapie}, welche den Spitäler Schaffhausen angegliedert ist. Die Klinik hat vom Kanton einen Versorgungsauftrag zur stationären, teilstationären und ambulanten Behandlung und Betreuung erwachsener psychisch kranker Menschen. Die Klinik verfügt insgesamt über 60 Betten in der Akutpsychiatrie und der Rehabilitation. Weitere 12 Plätze sind in der psychiatrischen Tagesklinik vorhanden. Das Gebiet für die Versorgung umfasst einerseits den gesamten Kanton Schaffhausen, sowie die angrenzenden Gebiete. Die Klinik arbeitet eng mit weiteren Kliniken der angrenzenden Kantone zusammen und nimmt bei Bedarf (wenn die angeschlossenen Kliniken keine Betten mehr haben) Patienten aus diesen auf.

Das Praktikum dauerte vom 3.1.2018 bis 30.3.2018 und umfasste insgesamt 433,5 Stunden, die bei einer Anstellung von 100\% bei 8,5 Stunden pro Tag geleistet wurden. 

Die Entlöhnung bei 100\% umfasste CHF 1'000.- (brutto) pro Monat und wurde per Ende Monat auf mein Lohnkonto überwiesen. Das Praktikum entsprach einer begrenzten Anstellung bei den Spitälern Schaffhausen, welches mit 5,5 Tagen Urlaub und einem vierten Teil des 13ten Monatslohn vergütet wurde. Für die Unterrichtstage an der ZHAW wurde ich vom Arbeitgeber freundlicherweise freigestellt. 

Für die Zeit in meinen Praktikum wurde ich einer Station fest zugewiesen. Diese Station hatte die Bezeichnung A2 und ist für Affektive-  und Persönlichkeitsstörungen im Akutbereich spezialisiert und besitzt eine Kapazität von rund 20 Betten. Des Weiteren verfügt die Station über einen Isolationsraum, der bei Bedarf klinikübergreifend eingesetzt werden kann. Die restlichen Zimmer sind in Zwei- und Dreibettzimmer aufgeteilt, wobei die Kapazität im Notfall mit sogenannten Überbetten erhöht werden kann. 

In den drei Monaten hatte ich die Möglichkeit neben der Station A2 weitere Stationen in Form einer Tageshospitation zu besuchen. Ich nahm von diesem Angebot gebrauch und verbrachte je einen Tag auf der Station für psychotische Störungen und in der Tagesklinik. 

%----------------------------------------------
\subsubsection{Praktikumsleitung (Person / Funktion)}
Im Psychiatriezentrum gab es bis zu meinem Austritt keine einzelne Stelle, die sich für die Praktikumsleitung verantwortlich sah. Die Leitung wurde unter verschiedenen Personen aufgeteilt: Für die zeitliche Planung und Auswahl der Praktikanten und Praktikantinnen zeichnete sich Frau Sybille Echte-Schnauber verantwortlich, Fachpsychologin Psychologischer Dienst Psychiatrie. Sie war meine erste Ansprechperson für die Bewerbung und ist für die Auswahl der Praktikanten und Praktikantinnen zuständig. Für den Inhalt des Praktikums zeichnete sich Herr Dr. biol. hum. Bernd Lehle verantwortlich. Aktuell leitender Psychologe der Tagesklinik, welcher die Rahmenordnung für die Praktikas im Bereich Psychologie vor einigen Jahren erstellte. 

Diese beiden oben erwähnten Personen waren für meine Auswahl und den groben Ablauf des Praktikums zuständig. Da alle Praktikantinnen und Praktikanten einer bestimmten Station angeschlossen werden, sind sie als angehende Psychologinnen und Psychologen automatisch dem zuständigen Oberarzt unterstellt. In meinem Fall war dies Herr Dr. med. Walter Brogiolo, Therapeutischer Bereichsleiter Akutpsychiatrie und Therapeutischer Leiter A2. Dieser wiederum war dem Chefarzt Herr PD Dr. med. Bernd Krämer unterstellt. Ich erwähne ihn aus dem einfachen Grund, da er bei Unstimmigkeiten im Verlauf eines Praktikums, welche nicht vom zuständigen Oberarzt, noch von der vermeintlichen Praktikumsleitung gelöst werden kann, intervenierend einzugreifen vermag. Zudem konnte ich Herrn Krämer als an den Praktikantinnen und Praktikanten interessierten Arzt kennenlernen, der auch gerne proaktiv Anregungen und Tipps zum individuellen Praktikum abgegeben hat.

%----------------------------------------------
\subsubsection{Angaben über Betreuung, Super- und Intervision}
Die eigentliche Praktikumsanleitung wurde von Frau M.Sc. Martina Piraccini übernommen. Psychologin in Ausbildung auf der Akutstation A2. Sie hatte die Funktion der ersten Ansprechperson und übernahm die Koordination der einzelnen Praktikumstätigkeiten. Diese Aufgabe hat sie mit der zuständigen Aktivierungstherapeutin, Frau Birgitta Busin, wahrgenommen, welche für meine Einbettung in die Pflege verantwortlich war. 

Das Pflegeteam führte monatlich Supervisionen durch, in denen es um teaminterne Themen ging. Weiter wurde monatlich eine Fallsupervision mit dem gesamte Team der Akutstation A2 (von der Pflege zum Oberarzt, über die Psychologinnen und Assistenzärztin der Station) durchgeführt. Diese Supervisionen wurden beide von einer externen Fachperson, Herr Thomas Disler, Eidgenössisch anerkannter Psychotherapeut und Diplomsupervisor, durchgeführt. An beiden Veranstaltungen durfte ich als Praktikant aktiv teilnehmen. 

Zudem kam ich in den Genuss eines spezifischen Coachings für Praktikanten bei Frau Ronit Rüttimann, Fachpsychologin psychologischer Dienst am Haus. Dieses Coaching nahm ich zu Beginn wöchentlich, anschliessend etwa jede zweite Woche wahr. Es diente dazu, Fragen und Probleme im Zusammenhang mit dem Praktikum zu besprechen.

Durch die ZHAW wurde eine Intervision aus Studierenden vorgeschrieben. Diese Intervision wurde von drei Mistudierenden und mir alle paar Wochen durchgeführt, in welchen wir unsere individuellen Anliegen und Fälle gegenseitig vorstellten und besprachen (\prettyref{sec:Intervision}). 

%----------------------------------------------
\subsubsection{Auflistung aller ausgeführter Funktionen}
Im Folgenden werde ich alle Funktionen auflisten, die ich in den drei Monaten durchführen konnte. In einem ersten Teil werde ich nur knapp auf diese eingehen und im zweiten Teil (\prettyref{sec:Auseinandersetzung}) auf ausgewählte Aspekte meines Praktikums näher und ausführlicher eingehen.
\begin{itemize}
        \item Diverse Hospitationen unter anderem bei Einzelgesprächen mit Patienten, Oberarzt-Visiten, Chefarzt-Visiten, Morgen-Rapporte, Somatik-Rapporte, Interdisziplinären-Rapporten, Elternrunden, welche vom KJPD organisiert wurden, Eintrittsgepsräche auf Station und Austrittsgespräche.
        \item Schreiben von Verlaufseinträgen im eingesetzten Kliniktool (Polypoint), welche sich aus unterschiedlichen Patienten-Begegnungen auf der Akutstation ergeben haben. Unter anderem Einträge aus Einzelgesprächen, Gruppen-Therapien, KJPD Infoveranstaltungen, Bewegungstherapien und Spaziergängen.
        \item Co-Leitung und einmalige Leitung der wöchentlichen Gruppentherapie.
        \item Führen von Einzelgesprächen ausgewählter Bezugspatienten.
        \item Co-Leitung der Psychoedukation, die vier Wochen andauerte.
        \item Teilnahme und Mitgestaltung der wöchentlichen Literaturgruppe.
        \item Teilnahme und Führung der Morgenrunde, die jeden Tag stattfand und unterschiedliche Themen beinhaltete.
        \item Einmalige Erstbetreuung eines ISO-Patienten (Alkoholtestung, Urinal, Zigarettenpause).
        \item Teilnahme an internen Weiterbildungsangeboten (\textit{AMDP Seminar, Elternschaft mit Borderline}).
        \item Begleiten von Patienten ausserhalb der klinik, wie den täglichen Spaziergängen, Begleitung zum Kiosk auf dem Areal, Begleitung zu den Therapien und Begleitung für Besorgungen zu den Patienten nach Hause.
        \item Aktive Teilnahme an ausgewählten Therapieangeboten, um einen besseren Einblick in den Patientenalltag zu erlangen: Musiktherapie, Ergotherapie, Bewegungstherapie, Entspannungsübungen und Weiterbildungsseminare.
    \end{itemize}