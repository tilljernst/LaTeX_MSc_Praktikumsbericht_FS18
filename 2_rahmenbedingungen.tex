\subsection{Rahmenbedingungen und Fakten (TBD: 1 bis 2 Seiten)} \label{sec:Rahmenbedingungen}

%----------------------------------------------
\subsubsection{Ort, Dauer und Entlöhnung} 
Das Praktikum im Rahmen der Ausbildung an der ZHAW zum Psychologen führte ich an der \textit{Klinik für Psychiatrie und Psychotherapie} aus, welche als psychiatrischen Dienst der Spitäler Schaffhausen angegliedert ist. Die Klinik hat vom Kanton Schaffhausen einen Versorgungsauftrag zur stationären, teilstationären und ambulanten Behandlung und Betreuung erwachsener psychisch Kranker. Die Klinik verfügt über 60 Betten in den Akutpsychiatrie und Rehabilitation. Zudem sind 12 Plätze in der psychiatrischen Tagesklinik vorhanden. Das Gebiet für Versorgung umfasst einerseits den gesamten Kanton Schaffhausen, sowie die angrenzenden Gebiete. Die Klinik arbeitet eng mit weiteren Kliniken des Kantons Zürich zusammen und nimmt bei Bedarf (wenn die angeschlossenen Kliniken keine Betten mehr haben) Patienten aus Zürich auf.

Das Praktikum dauerte vom 3.1.2018 bis und mit 30.3.2018 und umfasste insgesamt 433,5 Stunden zu 8,5 Stunden pro Tag bei einer Anstellung von 100\% (gemäss Vorgaben sind im Masterstudiengang 300h vorgesehen). 

Die Entlöhnung bei 100\% umfasste pro Monat CHF 1000.- (brutto) und wurde per Ende Monat auf das Lohnkonto überwiesen. Zusätzlich wurde ein Teil des 13ten Monatslohns Ende März überwiesen. Das Praktikum entsprach einer herkömmliche Anstellung am Spital. Aus diesem Grund wurden mit in dieser Zeit 5,5 Tage Ferien zur Verfügung gestellt. Zudem wurde ich an den in diesem Semester fälligen Unterrichtstagen an der ZHAW freigestellt. 

Für diesen drei Monaten wurde ich fest auf der Akutstation für affektive Störungen und Persönlichkeitsstörungen eingeplant. Diese Station hatte die Bezeichnung A2 und besitzt eine Kapazität von rund 20 Betten. Zudem verfügt diese Station über einen Isolationsraum, der bei Bedarf eingesetzt werden kann. Die restlichen Zimmer sind in Zwei- und Dreibettzimmer aufgeteilt, wobei die Kapazität im Notfall mit sogenannten Überbetten erhöht werden kann. 

In den drei Monaten hatte ich die Möglichkeit neben der Station A2 weitere Stationen in Form einer Tageshospitation zu besuchen. Ich nahm von diesem Angebot gebrauch und verbrachte je einen Tag auf der Station für psychotische Störungen und in der Tagesklinik. 

%----------------------------------------------
\subsubsection{Praktikumsleitung (Person / Funktion)}
Im Psychiatriezentrum gab es bis zu meinem Austritt kein Einzelperson, die für die Praktikumsleitung verantwortlich war. Die Leitung wurde anhand unterschiedlicher Teilbereichen unter verschiedenen Personen aufgeteilt: Für die zeitliche Planung und Auswahl der Praktikanten und Praktikantinnen zeichnete sich Frau Sybille Echte-Schnauber verantwortlich, Fachpsychologin Psychologischer Dienst Psychiatrie. Sie war Ansprechperson für die Bewerbung und war für die Auswahl der Psychologiepraktikantinnen und Praktikanten zuständig. Für den Inhalt des Praktikums zeichnete sich Herr Dr. biol. hum. Bernd Lehle verantwortlich. Aktuell leitender Psychologe der Tagesklinik. 

Diese beiden oben erwähnten Personen waren für die Auswahl und den groben Ablauf des Praktikums zuständig. Da alle Praktikantinnen und Praktikanten einer bestimmten Station angeschlossen sind, werden sie als angehende Psychologinnen und Psychologen automatisch dem zuständigen Oberarzt unterstellt. In meinem Fall war dies Herr Dr. med. Walter Brogiolo, Therapeutischer Bereichsleiter Akutpsychiatrie und Therapeutischer Leiter A2. Wie das in Kliniken Usus ist, möchte ich hier die oberste Instanz nicht unerwähnt lassen, welche durch den Chefarzt Herr PD Dr. med. Bernd Krämer abgedeckt wurde. Ich erwähne ihn aus dem einfachen Grund, da er bei Unstimmigkeiten im Verlauf eines Praktikums, welche nicht vom zuständigen Oberarzt, noch von der Praktikumsleitung gelöst werden kann, intervenierend einzugreifen vermag. Zudem konnte ich Herrn Krämer als an den Prkatikantinnen und Praktikanten interessierten Arzt kennenlernen, der auch gerne Anregungen und Tipps zum individuellen Praktikum abgegeben hat.

%----------------------------------------------
\subsubsection{Angaben über Betreuung, Super- und Intervision}
Die eigentliche Praktikumsanleitung wurde von Frau M.Sc. Martina Piraccini übernommen. Psychologin in Ausbildung auf der Akutstation A2 für affektive Störungen, auf welcher ich in den drei Monaten hauptsächlich angeschlossen war. Sie hatte die Funktion der ersten Ansprechperson und übernahm die Koordination der einzelnen Praktikumstätigkeiten. Diese Aufgabe hat sie mit der zuständigen Aktivierungstherapeutin, Frau Birgitta Busin, wahrgenommen, welche für meine Einbettung in die Pflege verantwortlich war. 

Einerseits führte das Pflegeteam monatlich eine Supervision durch. Zudem wurde monatlich eine Fallsupervision mit dem gesamte Team der Akutstation A2 (von der Pflege zum Oberarzt, über die Psychologinnen und Assistenzärztin der Station) durchgeführt. Diese Supervisionen wurden beide von einer externen Fachperson, Herr Thomas Disler, Eidgenössisch anerkannter Psychotherapeut und Diplomsupervisor, durchgeführt. An beiden Veranstaltungen durfte ich als Praktikant teilnehmen. 

Zudem wurde bei Bedarf ein spezifisch Coaching / Supervision für Praktikanten bei Frau Ronit Rüttimann, Fachpsychologin psychologischer Dienst, am Haus angeboten. Dieses Coaching nahm ich zu Beginn wöchentlich, danach etwas jede zweite Woche wahr. Dieses Coaching diente dazu, Fragen und Probleme im Zusammenhang mit dem Praktikum zu besprechen.

Durch die ZHAW wurde eine durch Studierenden geführte Intervision vorgeschrieben. Diese Intervision wurde von drei Mistudierenden und mir alle paar Wochen durchgeführt. Darin haben wir unsere individuellen Anliegen und Fälle gegenseitig vorgestellt und besprochen (\prettyref{sec:Intervision}). 

%----------------------------------------------
\subsubsection{Auflistung aller ausgeführter Funktionen}
Im Folgenden werde ich alle Funktionen auflisten, die ich in den drei Monaten durchführen konnte. In einem ersten Teil werde ich nur knapp auf diese eingehen und im zweiten Teil auf ausgewählte näher und ausführlicher eingehen.
\begin{itemize}
        \item Diverse Hospitationen unter anderem bei Einzelgesprächen mit Patienten, Oberarzt-Visiten, Chefarzt-Visiten, Morgen-Rapporte, Somatik-Rapporte, Interdisziplinären-Rapporten, Elternrunden, welche vom KJPD organisiert wurden, Eintrittsgepsräche auf Station und Austrittsgespräche.
        \item Schreiben von Verlaufseinträgen im eingesetzten Kliniktool (Polypoint), welche sich aus unterschiedlichen Patienten-Begegnungen auf der Akutstation ergeben haben, unter anderem aus Einzelgesprächen, Gruppen-Therapien, KJPD Infoveranstaltungen, Bewegungstherapien und Spaziergängen.
        \item Co-Leitung und einmalige Leitung der wöchentlichen Gruppentherapie.
        \item Führen von Einzelgesprächen ausgewählter Bezugspatienten.
        \item Co-Leitung der Psychoedukation, die vier Wochen andauerte.
        \item Teilnahme und Mitgestaltung der wöchentlichen Literaturgruppe.
        \item Teilnahme und Mitgestaltung der Morgenrunde, die jeden Tag stattfand.
        \item Einmalige Erstbetreuung eines ISO-Patienten (Alkoholtestung, Urinal, Zigarettenpause).
        \item Teilnahme an internen Weiterbildungsangeboten (\textit{AMDP Seminar, Elternschaft mit Borderline})
        \item Begleiten von Patienten wie den täglichen Spaziergängen, Begleitung zum Kiosk, Begleitung zu den Therapien und Begleitung für Besorgungen zu den Patienten nach Hause.
        \item Zudem aktive Teilnahme am Therapieprogramm, um einen besseren Einblick in den Patientenalltag zu erlangen: Musiktherapie, Ergotherapie, Bewegungstherapie, Entspannungsübungen und Weiterbildungsseminare.
    \end{itemize}